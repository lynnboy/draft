\rSec0[temp]{Templates}%
\indextext{template|(}

%gram: \rSec1[gram.temp]{Templates}
%gram:

\indextext{parameterized type|see{template}}%
\indextext{type generator|see{template}}

\pnum
A \term{template} defines a family of classes or functions.

\indextext{\idxcode{template}}%
%
\begin{bnf}
\nontermdef{template-declaration}\br
  \terminal{export}\opt \terminal{template <} template-parameter-list \terminal{>} declaration
\end{bnf}

\begin{bnf}
\nontermdef{template-parameter-list}\br
  template-parameter\br
  template-parameter-list \terminal{,} template-parameter
\end{bnf}

The
\grammarterm{declaration}
in a
\grammarterm{template-declaration}
shall

\begin{itemize}
\item declare or define a function or a class, or

\item define a member function, a member class or a static data member of a
class template or of a class nested within a class template, or

\item define a member template of a class or class template.
\end{itemize}

A \grammarterm{template-declaration} is a \grammarterm{declaration}.
\indextext{template!definition~of}A \grammarterm{template-declaration} is also a definition
if its \grammarterm{declaration} defines a function, a class, or a static data member.

\pnum
A
\grammarterm{template-declaration}
can appear only as a namespace scope or class scope declaration.
In a function template declaration, the
\grammarterm{declarator-id} shall be a \grammarterm{template-name}
(i.e., not a \grammarterm{template-id}).
\enternote
in a class template declaration, if the class name is a
\grammarterm{template-id},
the declaration declares a class template partial specialization~(\ref{temp.class.spec}).
\exitnote

\pnum
In a
\grammarterm{template-declaration},
explicit specialization, or explicit instantiation the
\grammarterm{init-declarator-list}
in the declaration shall contain at most one declarator.
When such a declaration is used to declare a class template,
no declarator is permitted.

\pnum
\indextext{template~name!linkage~of}%
A template name has linkage~(\ref{basic.link}).
A non-member
function template can have internal linkage; any other
template name shall have external linkage. Entities generated from
a template with internal linkage are
distinct from all entities generated in other translation units.
A template, a template explicit specialization~(\ref{temp.expl.spec}), or a class
template partial specialization shall not have C linkage.
If the linkage of one of these is something
other than C or \Cpp, the behavior is
\impldef{semantics of linkage specification on templates}.
Template definitions shall obey the one definition rule~(\ref{basic.def.odr}).
\enternote
default arguments for function templates and for member functions of
class templates are considered definitions for the purpose of template
instantiation~(\ref{temp.decls}) and must also obey the one definition rule.
\exitnote

\pnum
A class template shall not have the same name as any other
template, class, function, object, enumeration, enumerator, namespace, or
type in the same scope~(\ref{basic.scope}), except as specified in~(\ref{temp.class.spec}).
Except that a function template can be overloaded either by (non-template)
functions with the same name or by other function templates
with the same name~(\ref{temp.over}),
a template name declared in namespace scope or in class scope shall be unique
in that scope.

\pnum
A namespace-scope declaration or definition of a non-inline function
template, a non-inline member function template, a non-inline member
function of a class template or a static data member of a class template
may be preceded by the \tcode{export} keyword. If such a template is defined
in the same translation unit in which it is declared as exported, the
definition is considered to be \term{exported}. The first declaration of the
template containing the \tcode{export} keyword must not follow the definition.

\pnum
Declaring a class template exported is equivalent to declaring all of its
non-inline function members, static data members, member classes, member
class templates and non-inline function member templates which
are defined in that translation unit exported.

\pnum
Templates defined in an unnamed namespace shall not be exported. A template
shall be exported only once in a program. An implementation is not required
to diagnose a violation of this rule. A non-exported template must
be defined in every translation unit in
which it is implicitly instantiated~(\ref{temp.inst}), unless the
corresponding specialization is explicitly instantiated~(\ref{temp.explicit})
in some translation unit; no diagnostic is required.
\enternote
See also~\ref{temp.explicit}.
\exitnote
An exported template need only be declared (and not necessarily defined)
in a translation unit in which it is instantiated. A function template
declared both exported and inline is just inline and not exported.

\pnum
\enternote
an implementation may require that a translation unit containing the
definition of an exported template be compiled before any translation unit
containing an instantiation of that template.
\exitnote

\rSec1[temp.param]{Template parameters}

\pnum
The syntax for
\grammarterm{template-parameter}{s}
is:

\begin{bnf}
\nontermdef{template-parameter}\br
  type-parameter\br
  parameter-declaration
\end{bnf}

\begin{bnf}
\nontermdef{type-parameter}\br
  \terminal{class} identifier\opt\br
  \terminal{class} identifier\opt \terminal{=} type-id\br
  \terminal{typename} identifier\opt\br
  \terminal{typename} identifier\opt \terminal{=} type-id\br
  \terminal{template <} template-parameter-list \terminal{> class} identifier\opt\br
  \terminal{template <} template-parameter-list \terminal{> class} identifier\opt \terminal{=} id-expression
\end{bnf}

\pnum
There is no semantic difference between
\tcode{class}
and
\tcode{typename}
in a
\grammarterm{template-parameter}.
\tcode{typename}
followed by an
\grammarterm{unqualified-id}
names a template type parameter.
\tcode{typename}
followed by a
\grammarterm{qualified-id}
denotes the type in a non-type \footnote{Since template
\grammarterm{template-parameter}{s}
and template
\grammarterm{template-argument}{s}
are treated as types for descriptive purposes, the terms
\grammarterm{non-type parameter}
and
\grammarterm{non-type argument}
are used to refer to non-type, non-template parameters and arguments.}
\grammarterm{parameter-declaration}.
A storage class shall not be specified in a
\grammarterm{template-parameter}
declaration.
\enternote
a template parameter may be a class template.
For example,

\begin{codeblock}
template<class T> class myarray { /* ... */ };

template<class K, class V, template<class T> class C = myarray>
class Map {
	C<K> key;
	C<V> value;
	// ...
};
\end{codeblock}
\exitnoteb

\pnum
A
\grammarterm{type-parameter}
defines its
\grammarterm{identifier}
to be a
\grammarterm{typedef-name}
(if declared with
\tcode{class}
or
\tcode{typename})
or
\grammarterm{template-name}
(if declared with
\tcode{template})
in the scope of the template declaration.
\enternote
because of the name lookup rules, a
\grammarterm{template-parameter}
that could be interpreted as either a non-type
\grammarterm{template-parameter}
or a
\grammarterm{type-parameter}
(because its
\grammarterm{identifier}
is the name of an already existing class) is taken as a
\grammarterm{type-parameter}.
For example,

\begin{codeblock}
class T { /* ... */ };
int i;

template<class T, T i> void f(T t)
{
	T t1 = i;		// template-parameters \tcode{T} and \tcode{i}
	::T t2 = ::i;		// global namespace members \tcode{T} and \tcode{i}
}
\end{codeblock}

Here, the template
\tcode{f}
has a
\grammarterm{type-parameter}
called
\tcode{T},
rather than an unnamed non-type
\grammarterm{template-parameter}
of class
\tcode{T}.
\exitnote

\pnum
A non-type
\grammarterm{template-parameter}
shall have one of the following (optionally
\grammarterm{cv-qualified})
types:

\begin{itemize}
\item integral or enumeration type,

\item pointer to object or pointer to function,

\item reference to object or reference to function,

\item pointer to member.
\end{itemize}

\pnum
\enternote
other types are disallowed either explicitly below or implicitly by
the rules governing the form of
\grammarterm{template-argument}{s}~(\ref{temp.arg}).
\exitnote
The top-level
\grammarterm{cv-qualifiers}
on the
\grammarterm{template-parameter}
are ignored when determining its type.

\pnum
A non-type non-reference
\grammarterm{template-parameter}
is not an lvalue.
It shall not be assigned to or in any other way have its value changed.
A non-type non-reference
\grammarterm{template-parameter}
cannot have its address taken.
When a non-type non-reference
\grammarterm{template-parameter}
is used as an initializer for a reference, a temporary is always used.
\enterexample

\begin{codeblock}
template<const X& x, int i> void f()
{
	i++;                    // error: change of template-parameter value

	&x;                     // OK
	&i;                     // error: address of non-reference template-parameter

	int& ri = i;            // error: non-const reference bound to temporary
	const int& cri = i;     // OK: const reference bound to temporary
}
\end{codeblock}
\exitexampleb

\pnum
A non-type
\grammarterm{template-parameter}
shall not be declared to have floating point, class, or void type.
\enterexample

\begin{codeblock}
template<double d> class X;     // error
template<double* pd> class Y;   // OK
template<double& rd> class Z;   // OK
\end{codeblock}
\exitexampleb

\pnum
A non-type
\grammarterm{template-parameter}
of type ``array of
\tcode{T}''
or ``function returning
\tcode{T}''
is adjusted to be of type
``pointer to
\tcode{T}''
or ``pointer to function returning
\tcode{T}'',
respectively.
\enterexample

\begin{codeblock}
template<int *a>   struct R { /* ... */ };
template<int b[5]> struct S { /* ... */ };
int p;
R<&p> w;                        // OK
S<&p> x;                        // OK due to parameter adjustment
int v[5];
R<v> y;                         // OK due to implicit argument conversion
S<v> z;                         // OK due to both adjustment and conversion
\end{codeblock}
\exitexampleb

\pnum
A
\term{default template-argument}
is a
\grammarterm{template-argument}~(\ref{temp.arg}) specified after
\tcode{=}
in a
\grammarterm{template-parameter}.
A default
\grammarterm{tem\-plate-argument}
may be specified for any kind of
\grammarterm{template-parameter}
(type, non-type, template).
A default
\grammarterm{template-argument}
may be specified in a class template declaration or a class template
definition.
A default
\grammarterm{template-argument}
shall not be specified in a function template declaration or a function
template definition, nor in the
\grammarterm{template-parameter-list}
of the definition of a member of a class template.
A default
\grammarterm{template-argument}
shall not be specified in a friend template declaration.

\pnum
The set of default
\grammarterm{template-argument}{s}
available for use with a template
declaration or definition is obtained by merging the default arguments
from the definition (if in scope) and all declarations in scope in the
same way default function arguments are~(\ref{dcl.fct.default}).
\enterexample

\begin{codeblock}
template<class T1, class T2 = int> class A;
template<class T1 = int, class T2> class A;
\end{codeblock}

is equivalent to

\begin{codeblock}
template<class T1 = int, class T2 = int> class A;
\end{codeblock}
\exitexampleb

\pnum
If a
\grammarterm{template-parameter}
has a default
\grammarterm{template-argument},
all subsequent
\grammarterm{template-parameter}{s}
shall have a default
\grammarterm{template-argument}
supplied.
\enterexample

\begin{codeblock}
template<class T1 = int, class T2> class B;   // error
\end{codeblock}
\exitexampleb

\pnum
A
\grammarterm{template-parameter}
shall
not be given default arguments by two different declarations in the same scope.
\enterexample

\begin{codeblock}
template<class T = int> class X;
template<class T = int> class X { /*... */ }; // error
\end{codeblock}
\exitexampleb

\pnum
The scope of a \grammarterm{template-parameter} extends from its point of
declaration until the end of its template. In particular, a
\grammarterm{template-parameter} can be used in the declaration of
subsequent \grammarterm{template-parameter}{s} and their default arguments.
\enterexample

\begin{codeblock}
template<class T, T* p, class U = T> class X { /* ... */ };
template<class T> void f(T* p = new T);
\end{codeblock}

\exitexampleb

\pnum
A \grammarterm{template-parameter} shall not be used in its own default
argument.

\indextext{\idxcode{<}!template~and}%
\pnum
When parsing a default
\grammarterm{template-argument}
for a non-type
\grammarterm{template-parameter},
the first non-nested
\tcode{>}
is taken as the end of the
\grammarterm{template-parameter-list}
rather than a greater-than operator.
\enterexample

\begin{codeblock}
template<int i = 3 > 4 >        // syntax error
  class X { /* ... */ };

template<int i = (3 > 4) >      // OK
  class Y { /* ... */ };
\end{codeblock}
\exitexampleb

\rSec1[temp.names]{Names of template specializations}

\pnum
A template specialization~(\ref{temp.spec}) can be referred to by a
\grammarterm{template-id}:

\begin{bnf}
\nontermdef{template-id}\br
  template-name \terminal{<} template-argument-list\opt \terminal{>}
\end{bnf}

\begin{bnf}
\nontermdef{template-name}\br
  identifier
\end{bnf}

\begin{bnf}
\nontermdef{template-argument-list}\br
  template-argument\br
  template-argument-list \terminal{,} template-argument
\end{bnf}

\begin{bnf}
\nontermdef{template-argument}\br
  assignment-expression\br
  type-id\br
  id-expression
\end{bnf}

\enternote
the name lookup rules~(\ref{basic.lookup}) are used to associate the use of
a name with a template declaration;
that is, to identify a name as a
\grammarterm{template-name}.
\exitnote

\pnum
For a
\grammarterm{template-name}
to be explicitly qualified by the template arguments,
the name must be known to refer to a template.

\pnum
\indextext{\idxcode{<}!template~and}%
After name lookup~(\ref{basic.lookup}) finds that a name is a
\grammarterm{template-name},
if this name is followed by a
\tcode{<},
the
\tcode{<}
is always taken as the beginning of a
\grammarterm{template-argument-list}
and never as a name followed by the less-than operator.
When parsing a \grammarterm{template-id},
the first non-nested
\tcode{>}\footnote{A \tcode{>} that encloses the \grammarterm{type-id}
of a \tcode{dynamic_cast}, \tcode{static_cast}, \tcode{reinterpret_cast}
or \tcode{const_cast}, or which encloses the \grammarterm{template-argument}{s}
of a subsequent \grammarterm{template-id}, is considered nested for the purpose
of this description.
}
is taken as the end of the \grammarterm{template-argument-list}
rather than a greater-than operator.
\enterexample

\begin{codeblock}
template<int i> class X @\tcode{\{ /* ... */ \};}@

X< 1>2 >	x1;             // syntax error
X<(1>2)>	x2;             // OK

template<class T> class Y @\tcode{\{ /* ... */ \};}@
Y< X<1> >	x3;             // OK
Y<X<6>> 1> >	x4;             // OK: \tcode{Y< X< (6\shr1) > >}
\end{codeblock}
\exitexampleb

\pnum
When the name of a member template specialization appears after
\tcode{.}
or
\tcode{->}
in a
\grammarterm{postfix-expression},
or after
\grammarterm{nested-name-specifier}
in a
\grammarterm{qualified-id},
and the
\grammarterm{postfix-expression}
or
\grammarterm{qualified-id}
explicitly depends on a
\grammarterm{template-parameter}~(\ref{temp.dep}),
the member template name must be prefixed by the keyword
\tcode{template}.
Otherwise the name is assumed to name a non-template.
\enterexample

\begin{codeblock}
class X {
public:
	template<size_t> X* alloc();
	template<size_t> static X* adjust();
};
template<class T> void f(T* p)
{
	T* p1 = p->alloc<200>();
		// ill-formed: \tcode{<} means less than

	T* p2 = p->template alloc<200>();
		// OK: \tcode{<} starts template argument list

	T::adjust<100>();
		// ill-formed: \tcode{<} means less than

	T::template adjust<100>();
		// OK: \tcode{<} starts template argument list
}
\end{codeblock}
\exitexampleb

\pnum
If a name prefixed by the keyword
\tcode{template}
is not the name of a member template, the program is ill-formed.
\enternote
the keyword
\tcode{template}
may not be applied to non-template members of class templates.
\exitnote
Furthermore, names of member templates shall not be prefixed by the keyword
\tcode{template}
if the \grammarterm{postfix-expression} or \grammarterm{qualified-id}
does not appear in the scope of a template.
\enternote
just as is the case with the
\tcode{typename}
prefix, the
\tcode{template}
prefix is allowed in cases where it is not strictly
necessary; i.e., when the expression on the left of the
\tcode{->}
or
\tcode{.}, or the \grammarterm{nested-name-specifier}
is not dependent on a
\grammarterm{template-parameter}.
\exitnote

\pnum
\indextext{specialization!class template}%
A
\grammarterm{template-id}
that names a class template specialization is a
\grammarterm{class-name}
(clause~\ref{class}).

\rSec1[temp.arg]{Template arguments}

\pnum
\indextext{argument!template}%
There are three forms of
\grammarterm{template-argument},
corresponding to the three forms of
\grammarterm{template-parameter}:
type, non-type and template.
The type and form of each
\grammarterm{template-argument}
specified in a
\grammarterm{template-id}
shall match the type and form specified for the corresponding
parameter declared by the template in its
\grammarterm{template-parameter-list}.
\enterexample

\begin{codeblock}
template<class T> class Array {
	T* v;
	int sz;
public:
	explicit Array(int);
	T& operator[](int);
	T& elem(int i) { return v[i]; }
	// ...
};

Array<int> v1(20);
typedef complex<double> dcomplex;	// \tcode{complex} is a standard
					// library template
Array<dcomplex> v2(30);
Array<dcomplex> v3(40);

void bar() {
	v1[3] = 7;
	v2[3] = v3.elem(4) = dcomplex(7,8);
}
\end{codeblock}
\exitexampleb

\pnum
In a
\grammarterm{template-argument},
an ambiguity between a
\grammarterm{type-id}
and an expression is resolved to a
\grammarterm{type-id},
regardless of the form of the corresponding
\grammarterm{template-parameter}.\footnote{There is no such ambiguity in a default
\grammarterm{template-argument}
because the form of the
\grammarterm{template-parameter}
determines the allowable forms of the
\grammarterm{template-argument}.}
\enterexample

\begin{codeblock}
template<class T> void f();
template<int I> void f();

void g()
{
	f<int()>();		// \tcode{int()} is a type-id: call the first \tcode{f()}
}
\end{codeblock}
\exitexampleb

\pnum
The name of a
\grammarterm{template-argument}
shall be accessible at the point where it is used as a
\grammarterm{template-argument}.
\enternote
if the name of the
\grammarterm{template-argument}
is accessible at the point where it is used as a
\grammarterm{template-argument},
there is no further access restriction in the resulting instantiation where the
corresponding
\grammarterm{template-parameter}
name is used.
\exitnote
\enterexample

\begin{codeblock}
template<class T> class X {
	static T t;
};

class Y {
private:
	struct S { /* ... */ };
	X<S> x;			// OK: \tcode{S} is accessible
				// \tcode{X<Y::S>} has a static member of type \tcode{Y::S}
				// OK: even though \tcode{Y::S} is private
};

X<Y::S> y;			// error: \tcode{S} not accessible
\end{codeblock}
\exitexampleb
For a
\grammarterm{template-argument}
of class type, the template
definition has no special access rights to the inaccessible
members of the template-argument type.

\pnum
When default
\grammarterm{template-argument}{s}
are used, a
\grammarterm{template-argument}
list can be empty.
In that case the empty
\tcode{<>}
brackets shall still be used as the
\grammarterm{template-argument-list.}
\enterexample

\begin{codeblock}
template<class T = char> class String;
String<>* p;                    // OK: \tcode{String<char>}
String* q;                      // syntax error
\end{codeblock}
\exitexampleb

\pnum
An explicit destructor call~(\ref{class.dtor}) for an object that has a type
that is a class template specialization may explicitly specify the
\grammarterm{template-argument}{s}.
\enterexample

\begin{codeblock}
template<class T> struct A {
	~A();
};
void f(A<int>* p, A<int>* q) {
	p->A<int>::~A();        // OK: destructor call
	q->A<int>::~A<int>();   // OK: destructor call
}
\end{codeblock}
\exitexampleb

\pnum
If the use of a
\grammarterm{template-argument}
gives rise to an ill-formed construct in the instantiation of a
template specialization, the program is ill-formed.

\pnum
When the template in a
\grammarterm{template-id}
is an overloaded function template, both non-template functions in the overload
set and function templates in the overload set for
which the
\grammarterm{template-argument}{s}
do not match the
\grammarterm{template-parameter}{s}
are ignored.
If none of the function templates have matching
\grammarterm{template-parameter}{s},
the program is ill-formed.

\rSec2[temp.arg.type]{Template type arguments}

\pnum
A
\grammarterm{template-argument}
for a
\grammarterm{template-parameter}
which is a type
shall be a
\grammarterm{type-id}.

\pnum
A local type, a type with no linkage, an unnamed type or a type compounded
from any of these types shall not be used as a \grammarterm{template-argument}
for a template \grammarterm{type-parameter}.
\enterexample
\begin{codeblock}
template <class T> class X { /* ... */ };
void f()
{
    struct S { /* ... */ };

    X<S> x3;			// error: local type used as \grammarterm{template-argument}
    X<S*> x4;			// error: pointer to local type used as \grammarterm{template-argument}
}
\end{codeblock}
\exitexampleb
\enternote
a template type argument may be an incomplete type~(\ref{basic.types}).
\exitnote

\pnum
If a declaration acquires a function type through a type dependent on a
\grammarterm{template-parameter}
and this causes a declaration that does not use the
syntactic form of a function declarator to have function type,
the program is ill-formed.
\enterexample

\begin{codeblock}
template<class T> struct A {
	static T t;
};
typedef int function();
A<function> a;                  // ill-formed: would declare \tcode{A<function>::t}
                                // as a static member function
\end{codeblock}
\exitexampleb

\rSec2[temp.arg.nontype]{Template non-type arguments}

\pnum
A
\grammarterm{template-argument}
for a non-type, non-template
\grammarterm{template-parameter}
shall be one of:
\begin{itemize}
\item
an integral \grammarterm{constant-expression} of integral or enumeration type;
or
\item
the name of a non-type
\grammarterm{template-parameter};
or
\item
the address of an object
or function with external linkage,
including function templates and function
\grammarterm{template-id}{s}
but
excluding non-static class members, expressed as
\tcode{\&}
\grammarterm{id-expression}{}, where
the
\tcode{\&}
is optional if the name refers to a function or
array, or if the corresponding
\grammarterm{template-parameter}
is a reference;
or
\item
a pointer to member expressed as described in~\ref{expr.unary.op}.
\end{itemize}

\pnum
\enternote
A string literal~(\ref{lex.string})
does not satisfy the requirements of any of these
categories and thus is not an acceptable
\grammarterm{template-argument}.
\enterexample

\begin{codeblock}
template<class T, char* p> class X {
	// ...
	X();
	X(const char* q) { /* ... */ }
};

X<int,"Studebaker"> x1;		// error: string literal as template-argument

char p[] = "Vivisectionist";
X<int,p> x2;                    // OK
\end{codeblock}
\exitexampleb
\exitnoteb

\pnum
\enternote
Addresses of array elements and names or addresses of non-static class
members are not acceptable
\grammarterm{template-argument}{s}.
\enterexample

\begin{codeblock}
template<int* p> class X { };

int a[10];
struct S { int m; static int s; } s;

X<&a[2]> x3;                    // error: address of array element
X<&s.m> x4;                     // error: address of non-static member
X<&s.s> x5;                     // error: \tcode{\&S::s} must be used
X<&S::s> x6;                    // OK: address of static member
\end{codeblock}
\exitexampleb
\exitnoteb

\pnum
\enternote
Temporaries, unnamed lvalues, and named lvalues that do not have external
linkage are not acceptable
\grammarterm{template-argument}{s}
when the corresponding
\grammarterm{template-parameter}
has reference type.
\enterexample

\begin{codeblock}
template<const int& CRI> struct B { /* ... */ };

B<1> b2;                        // error: temporary would be required for template argument

int c = 1;
B<c> b1;                        // OK
\end{codeblock}
\exitexampleb
\exitnoteb

\pnum
The following conversions are performed on each expression used as a non-type
\grammarterm{template-argument}.
If a non-type
\grammarterm{template-argument}
cannot be converted to the type of the corresponding
\grammarterm{template-parameter}
then the program is ill-formed.

\begin{itemize}
\item
for a non-type
\grammarterm{template-parameter}
of integral or enumeration type, integral promotions~(\ref{conv.prom}) and
integral conversions~(\ref{conv.integral}) are applied.
\item
for a non-type
\grammarterm{template-parameter}
of type pointer to object,
qualification conversions~(\ref{conv.qual}) and
the array-to-pointer conversion~(\ref{conv.array}) are applied.
\enternote
In particular, neither the null pointer conversion~(\ref{conv.ptr}) nor the
derived-to-base conversion~(\ref{conv.ptr}) are applied.
Although
\tcode{0}
is a valid
\grammarterm{template-argument}
for a non-type
\grammarterm{template-parameter}
of integral type, it is not a valid
\grammarterm{template-argument}
for a non-type
\grammarterm{template-parameter}
of pointer type.
\exitnote
\item
For a non-type
\grammarterm{template-parameter}
of type reference to object,
no conversions apply.
The type referred to by the reference may be more cv-qualified than the
(otherwise identical) type of the
\grammarterm{template-argument}.
The
\grammarterm{template-parameter}
is bound directly to the
\grammarterm{template-argument},
which must be an lvalue.
\item
For a non-type
\grammarterm{template-parameter}
of type pointer to function, only the function-to-pointer conversion~(\ref{conv.func})
is applied.
If the
\grammarterm{template-argument}
represents a set of overloaded functions (or a pointer to such), the matching
function is selected from the set~(\ref{over.over}).
\item
For a non-type
\grammarterm{template-parameter}
of type reference to function, no conversions apply.
If the
\grammarterm{template-argument}
represents a set of overloaded functions, the matching function is selected
from the set~(\ref{over.over}).
\item
For a non-type
\grammarterm{template-parameter}
of type pointer to member function, no conversions apply.
If the
\grammarterm{template-argument}
represents a set of overloaded member functions, the matching
member function is selected from the set~(\ref{over.over}).
\item
For a non-type
\grammarterm{template-parameter}
of type pointer to data member,
qualification conversions~(\ref{conv.qual})
are applied.
\end{itemize}

\enterexample
\begin{codeblock}
template<const int* pci> struct X { /* ... */ };
int ai[10];
X<ai> xi;                       // array to pointer and qualification conversions

struct Y { /* ... */ };
template<const Y& b> struct Z { /* ... */ };
Y y;
Z<y> z;                         // no conversion, but note extra cv-qualification

template<int (&pa)[5]> struct W { /* ... */ };
int b[5];
W<b> w;                         // no conversion

void f(char);
void f(int);

template<void (*pf)(int)> struct A { /* ... */ };

A<&f> a;                        // selects \tcode{f(int)}
\end{codeblock}
\exitexampleb

\rSec2[temp.arg.template]{Template template arguments}

\pnum
A
\grammarterm{template-argument}
for a template
\grammarterm{template-parameter}
shall be the name of a class template, expressed as
\grammarterm{id-expression}.
Only primary class templates are considered when matching the template template
argument with the corresponding parameter; partial specializations are not
considered even if their parameter lists match that of the template template
parameter.

\pnum
Any partial specializations~(\ref{temp.class.spec}) associated with the
primary class template are considered when a specialization based on the
template
\grammarterm{template-parameter}
is instantiated.
If a specialization is not visible at the point of instantiation,
and it would have been selected had it been visible, the program is ill-formed;
no diagnostic is required.
\enterexample

\begin{codeblock}
template<class T> class A {     // primary template
	int x;
};
template<class T> class A<T*> { // partial specialization
	long x;
};
template<template<class U> class V> class C {
	V<int>  y;
	V<int*> z;
};
C<A> c;                         // \tcode{V<int>} within \tcode{C<A>} uses the primary template,
                                // so \tcode{c.y.x} has type \tcode{int}
                                // \tcode{V<int*>} within \tcode{C<A>} uses the partial specialization,
                                // so \tcode{c.z.x} has type \tcode{long}
\end{codeblock}
\exitexampleb

\rSec1[temp.type]{Type equivalence}

\pnum
\indextext{equivalence!template type}%
Two \grammarterm{template-id}{s} refer to the same class or function if
their template names are identical, they refer to the same template,
their type \grammarterm{template-argument}{s} are the same type,
their non-type \grammarterm{template-argument}{s} of
integral or enumeration type have identical values,
their non-type \grammarterm{template-argument}{s} of
pointer or reference type refer to the same external object or function, and
their template \grammarterm{template-argument}{s} refer
to the same template.
\enterexample

\begin{codeblock}
template<class E, int size> class buffer { /* ... */ };
buffer<char,2*512> x;
buffer<char,1024> y;
\end{codeblock}

declares
\tcode{x}
and
\tcode{y}
to be of the same type, and

\begin{codeblock}
template<class T, void(*err_fct)()> class list { /* ... */ };
list<int,&error_handler1> x1;
list<int,&error_handler2> x2;
list<int,&error_handler2> x3;
list<char,&error_handler2> x4;
\end{codeblock}

declares
\tcode{x2}
and
\tcode{x3}
to be of the same type.
Their type differs from the types of
\tcode{x1}
and
\tcode{x4}.
\exitexample

\rSec1[temp.decls]{Template declarations}

\pnum
A
\grammarterm{template-id},
that is, the
\grammarterm{template-name}
followed by a
\grammarterm{template-argument-list}
shall not be specified in the declaration of a primary template declaration.
\enterexample

\begin{codeblock}
template<class T1, class T2, int I> class A<T1, T2, I> { };     // error
template<class T1, int I> void sort<T1, I>(T1 data[I]);         // error
\end{codeblock}
\exitexampleb
\enternote
however, this syntax is allowed in class template partial specializations~(\ref{temp.class.spec}).
\exitnote

\pnum
For purposes of name lookup and instantiation,
default arguments of function templates and default arguments of
member functions of class templates are considered definitions;
each default argument is a separate definition which is unrelated to
the function template definition or to any other default arguments.

\rSec2[temp.class]{Class templates}

\pnum
A class
\term{template}
defines the layout and operations
for an unbounded set of related types.
\enterexample
a single class template
\tcode{List}
might provide a common definition for
list of
\tcode{int},
list of
\tcode{float},
and list of pointers to
\tcode{Shape}s.
\exitexample

\pnum
\enterexample
An array class template might be declared like this:

\begin{codeblock}
template<class T> class Array {
    T* v;
    int sz;
public:
    explicit Array(int);
    T& operator[](int);
    T& elem(int i) { return v[i]; }
    // ...
};
\end{codeblock}

The prefix
\tcode{template}
\tcode{<class}
\tcode{T>}
specifies that a template is being declared and that a
\grammarterm{type-name}
\tcode{T}
will be used in the declaration.
In other words,
\tcode{Array}
is a parameterized type with
\tcode{T}
as its parameter.
\exitexample

\pnum
When a member function, a member class, a static data member or
a member template of a class
template is defined outside of the class template definition,
the member definition is defined as a template definition in which the
\grammarterm{template-parameter}{s}
are those of the class template.
The names of the template parameters used in the definition of the member may
be different from the template parameter names used in the class
template definition.
The template argument list following the class template name in the member
definition shall name the parameters in the same order as the one used in
the template parameter list of the member.
\enterexample

\begin{codeblock}
template<class T1, class T2> struct A {
    void f1();
    void f2();
};

template<class T2, class T1> void A<T2,T1>::f1() { }    // OK
template<class T2, class T1> void A<T1,T2>::f2() { }    // error
\end{codeblock}

\exitexampleb

\pnum
In a redeclaration, partial
specialization,
explicit specialization or explicit
instantiation of a class template, the
\grammarterm{class-key}
shall agree in kind with the original class template declaration~(\ref{dcl.type.elab}).

\rSec3[temp.mem.func]{Member functions of class templates}

\pnum
\indextext{template!member~function}%
A member function
of a class template
may be defined outside of the class
template definition in which it is declared.
\enterexample

\begin{codeblock}
template<class T> class Array {
    T* v;
    int sz;
public:
    explicit Array(int);
    T& operator[](int);
    T& elem(int i) { return v[i]; }
    // ...
};
\end{codeblock}

declares three function templates.
The subscript function might be defined like this:

\begin{codeblock}
template<class T> T& Array<T>::operator[](int i)
{
    if (i<0 || sz<=i) error("Array: range error");
    return v[i];
}
\end{codeblock}
\exitexampleb

\pnum
The
\grammarterm{template-argument}{s}
for a member function of a class template are determined by the
\grammarterm{template-argument}{s}
of the type of the object for which the member function is called.
\enterexample
the
\grammarterm{template-argument}
for
\tcode{Array<T>\,::\,op\-er\-a\-tor\,[]\,()}
will be determined by the
\tcode{Array}
to which the subscripting operation is applied.

\begin{codeblock}
Array<int> v1(20);
Array<dcomplex> v2(30);

v1[3] = 7;                      // \tcode{Array<int>::operator[]()}
v2[3] = dcomplex(7,8);          // \tcode{Array<dcomplex>::operator[]()}
\end{codeblock}
\exitexampleb

\rSec3[temp.mem.class]{Member classes of class templates}

\pnum
A class member of a class template may be defined outside the class template
definition in which it is declared.
\enternote
the class member must be defined before its first use that requires
an instantiation~(\ref{temp.inst}).
For example,

\begin{codeblock}
template<class T> struct A {
	class B;
};
A<int>::B* b1;                  // OK: requires \tcode{A} to be defined but not \tcode{A::B}
template<class T> class A<T>::B { };
A<int>::B  b2;                  // OK: requires \tcode{A::B} to be defined
\end{codeblock}
\exitnoteb

\rSec3[temp.static]{Static data members of class templates}

\pnum
\indextext{member!template~and \tcode{static}}%
A definition for a static data member may be provided in a
namespace scope enclosing the definition of the static member's class template.
\enterexample

\begin{codeblock}
template<class T> class X {
	static T s;
};
template<class T> T X<T>::s = 0;
\end{codeblock}
\exitexampleb

\rSec2[temp.mem]{Member templates}

\pnum
A template can be declared within a class or class template; such a template
is called a member template.
A member template can be defined within or outside its class definition or
class template definition.
A member template of a class template that is defined outside of its class
template definition shall be specified with the
\grammarterm{template-parameter}{s}
of the class template followed by the
\grammarterm{template-parameter}{s}
of the member template.
\enterexample

\begin{codeblock}
template<class T> class string {
public:
	template<class T2> int compare(const T2&);
	template<class T2> string(const string<T2>& s) { /* ... */ }
	// ...
};

template<class T> template<class T2> int string<T>::compare(const T2& s)
{
	// ...
}
\end{codeblock}
\exitexampleb

\pnum
A local class shall not have member templates.
Access control rules (clause~\ref{class.access})
apply to member template names.
A destructor shall not be a member
template.
A normal (non-template) member function with a given name
and type and a member function template of the same name, which could be
used to generate a specialization of the same type, can both be
declared in a class.
When both exist, a use of that name and type refers to the
non-template member unless an explicit template argument list is supplied.
\enterexample

\begin{codeblock}
template <class T> struct A {
	void f(int);
	template <class T2> void f(T2);
};

template <> void A<int>::f(int) { }                     // non-template member
template <> template <> void A<int>::f<>(int) { }       // template member

int main()
{
	A<char> ac;
	ac.f(1);		// non-template
	ac.f('c');		// template
	ac.f<>(1);		// template
}
\end{codeblock}
\exitexampleb

\pnum
A member function template shall not be virtual.
\enterexample

\begin{codeblock}
template <class T> struct AA {
	template <class C> virtual void g(C);   // error
	virtual void f();                       // OK
};
\end{codeblock}
\exitexampleb

\pnum
A specialization of
a member function template does not override a virtual function from a
base class.
\enterexample

\begin{codeblock}
class B {
	virtual void f(int);
};

class D : public B {
	template <class T> void f(T);	// does not override \tcode{B::f(int)}
	void f(int i) { f<>(i); }	// overriding function that calls
					// the template instantiation
};
\end{codeblock}
\exitexampleb

\pnum
A specialization of a template conversion function is referenced in
the same way as a non-template conversion function that converts to
the same type.
\enterexample

\begin{codeblock}
struct A {
	template <class T> operator T*();
};
template <class T> A::operator T*(){ return 0; }
template <> A::operator char*(){ return 0; }    // specialization
template A::operator void*();                   // explicit instantiation

int main()
{
	A      a;
	int*   ip;

	ip = a.operator int*();		// explicit call to template operator
					// \tcode{A::operator int*()}
}
\end{codeblock}
\exitexampleb
\enternote
because the explicit template argument list follows the function template
name, and because conversion member function templates and constructor
member function templates are called without using a function name,
there is no way to provide an explicit template argument list for these
function templates.
\exitnote

\pnum
A specialization of a template conversion function is not found by name
lookup. Instead, any template conversion functions visible in the
context of the use are considered. For each such operator, if argument
deduction succeeds~(\ref{temp.deduct.conv}), the resulting specialization is
used as if found by name lookup.

\pnum
A \grammarterm{using-declaration} in a derived class cannot refer to a specialization
of a template conversion function in a base class.

\pnum
Overload resolution~(\ref{over.ics.rank}) and partial
ordering~(\ref{temp.func.order}) are used to select the best conversion function
among multiple template conversion functions
and/or non-template conversion functions.

\rSec2[temp.friend]{Friends}

\pnum
\indextext{friend!template~and}%
A friend of a class or class template can be a function template or
class template, a specialization of a function template or class
template, or an ordinary (nontemplate) function or class.
For a friend function declaration that is not a template declaration:

\begin{itemize}
\item
if the name of the friend is a qualified or unqualified \grammarterm{template-id},
the friend declaration refers to a specialization of a function
template, otherwise
\item
if the name of the friend is a \grammarterm{qualified-id} and a matching nontemplate
function is found in the specified class or namespace, the friend
declaration refers to that function, otherwise,
\item
if the name of the friend is a \grammarterm{qualified-id} and a matching
specialization of a function template is found in the specified class
or namespace, the friend declaration refers to
that function template specialization, otherwise,
\item
the name shall be an \grammarterm{unqualified-id} that declares (or redeclares) an
ordinary (nontemplate) function.
\end{itemize}

\enterexample
\begin{codeblock}
template<class T> class task;
template<class T> task<T>* preempt(task<T>*);

template<class T> class task {
	// ...
	friend void next_time();
	friend void process(task<T>*);
	friend task<T>* preempt<T>(task<T>*);
	template<class C> friend int func(C);

	friend class task<int>;
	template<class P> friend class frd;
	// ...
};
\end{codeblock}

Here,
each specialization of the
\tcode{task}
class template has the function
\tcode{next_time}
as a friend;
because
\tcode{process}
does not have explicit
\grammarterm{template-argument}{s},
each specialization of the
\tcode{task}
class template has an appropriately typed function
\tcode{process}
as a friend, and this friend is not a function template specialization;
because the friend
\tcode{preempt}
has an explicit
\grammarterm{template-argument}
\tcode{<T>},
each specialization of the
\tcode{task}
class template has the appropriate specialization of the function
template
\tcode{preempt}
as a friend;
and each specialization of the
\tcode{task}
class template has all specializations of the function template
\tcode{func}
as friends.
Similarly,
each specialization of the
\tcode{task}
class template has the class template specialization
\tcode{task<int>}
as a friend, and has all specializations of the class template
\tcode{frd}
as friends.
\exitexampleb

\pnum
A friend function declaration that is not a template declaration and in
which the name of the friend is an unqualified \grammarterm{template-id}
shall refer to a specialization of a function template declared in the
nearest enclosing namespace scope.
\enterexample

\begin{codeblock}
namespace N {
	template <class T> void f(T);
	void g(int);
	namespace M {
		template <class T> void h(T);
		template <class T> void i(T);
		struct A {
			friend void f<>(int);	// ill-formed -- \tcode{N::f}
			friend void h<>(int);	// OK -- \tcode{M::h}
			friend void g(int);	// OK -- new decl of \tcode{M::g}
			template <class T> void i(T);
			friend void i<>(int);	// ill-formed -- \tcode{A::i}
		};
	}
}
\end{codeblock}

\exitexampleb

\pnum
A friend template may be declared within a class or class template.
A friend function template may be defined within a class or class
template, but a friend class template may not be defined in a class
or class template.
In these cases, all specializations of the friend class or friend function
template are friends of the class or class template granting friendship.
\enterexample

\begin{codeblock}
class A {
	template<class T> friend class B;       // OK
	template<class T> friend void f(T)@\tcode{\{ /* ... */ \}}@	    // OK
};
\end{codeblock}
\exitexampleb

\pnum
A template friend declaration specifies that all specializations of that
template, whether they are implicitly instantiated~(\ref{temp.inst}), partially
specialized~(\ref{temp.class.spec}) or explicitly specialized~(\ref{temp.expl.spec}),
are friends of the class containing the template friend declaration.
\enterexample

\begin{codeblock}
class X {
	template<class T> friend struct A;
	class Y { };
};

template<class T> struct A { X::Y ab; };        // OK
template<class T> struct A<T*> { X::Y ab; };    // OK
\end{codeblock}
\exitexampleb

\pnum
When a function is defined in a friend function declaration in a class
template, the function is defined at each instantiation of the class
template. The function is defined even if it is never used. The
same restrictions on multiple declarations and definitions which apply
to non-template function declarations and definitions also apply to
these implicit definitions.
\enternote
if the function definition is ill-formed for a given specialization of
the enclosing class template, the program is ill-formed even if the
function is never used.
\exitnote

\pnum
A member of a class template may be declared to be a friend of a
non-template class.
In this case, the corresponding member of every specialization of
the class template is a friend of the class granting friendship.
\enterexample

\begin{codeblock}
template<class T> struct A {
	struct B { };
	void f();
};

class C {
	template<class T> friend struct A<T>::B;
	template<class T> friend void A<T>::f();
};
\end{codeblock}
\exitexampleb

\pnum
\enternote
a friend declaration may first declare a member of an enclosing namespace scope~(\ref{temp.inject}).
\exitnote

\pnum
A friend template shall not be declared in a local class.

\pnum
Friend declarations shall not declare partial specializations.
\enterexample

\begin{codeblock}
template<class T> class A { };
class X {
	template<class T> friend class A<T*>;	// error
};
\end{codeblock}
\exitexampleb

\pnum
When a friend declaration refers to a specialization of a function
template, the function parameter declarations shall not include
default arguments, nor shall the inline specifier be used in such a
declaration.

\rSec2[temp.class.spec]{Class template partial specializations}

\pnum
\indextext{specialization!class template partial}%
\indextext{template!primary}%
A
\term{primary}
class template declaration is one in which the class template name is an
identifier.
A template declaration in which the class template name is a
\grammarterm{template-id},
is a
\term{partial specialization}
of the class template named in the
\grammarterm{template-id}.
A partial specialization of a class template provides an alternative definition
of the template that is used instead of the primary definition when the
arguments in a specialization match those given in the partial
specialization~(\ref{temp.class.spec.match}).
The primary template shall be declared before any specializations of
that template. If a template is partially specialized then that
partial specialization shall be declared before the first use of that partial
specialization that would cause an implicit instantiation to take place,
in every translation unit in which such a use
occurs; no diagnostic is required.

\pnum
When a partial specialization is used within the instantiation of an
exported template, and the unspecialized template name is non-dependent
in the exported template, a declaration of the partial specialization
must be declared before the definition of the exported template, in the
translation unit containing that definition. A similar restriction applies
to explicit specialization; see~\ref{temp.spec}.

\pnum
Each class template partial specialization is a distinct template and
definitions shall be provided for the members of a template partial
specialization~(\ref{temp.class.spec.mfunc}).

\pnum
\enterexample
\begin{codeblock}
template<class T1, class T2, int I> class A             { };    // \#1
template<class T, int I>            class A<T, T*, I>   { };    // \#2
template<class T1, class T2, int I> class A<T1*, T2, I> { };    // \#3
template<class T>                   class A<int, T*, 5> { };    // \#4
template<class T1, class T2, int I> class A<T1, T2*, I> { };    // \#5
\end{codeblock}

The first declaration declares the primary (unspecialized) class template.
The second and subsequent declarations declare partial specializations of
the primary template.
\exitexample

\pnum
The template parameters are specified in the angle bracket enclosed list
that immediately follows the keyword
\tcode{tem\-plate}.
For partial specializations, the template argument list is explicitly
written immediately following the class template name.
For primary templates, this list is implicitly described by the
template parameter list.
Specifically, the order of the template arguments is the sequence in
which they appear in the template parameter list.
\enterexample
the template argument list for the primary template in the example
above is
\tcode{<T1,}
\tcode{T2,}
\tcode{I>}.
\exitexample
\enternote
the template argument list shall not be specified in the primary template
declaration.
For example,

\begin{codeblock}
template<class T1, class T2, int I> class A<T1, T2, I>  { };    // error
\end{codeblock}
\exitnoteb

\pnum
A class template partial specialization may be declared or redeclared in any
namespace scope in which its definition may be defined~(\ref{temp.class} and~\ref{temp.mem}).
\enterexample

\begin{codeblock}
template<class T> struct A {
	class C {
		template<class T2> struct B { };
	};
};

// partial specialization of \tcode{A<T>::C::B<T2>}
template<class T> template<class T2>
	struct A<T>::C::B<T2*> { };

A<short>::C::B<int*> absip;     // uses partial specialization
\end{codeblock}
\exitexampleb

\pnum
Partial specialization declarations themselves are not found by name lookup.
Rather, when the primary template name is used, any previously declared partial
specializations of the primary template are also considered.
One consequence is
that a
\grammarterm{using-declaration}
which refers to a class template does not restrict the set of partial specializations
which may be found through the
\grammarterm{using-declaration}.
\enterexample

\begin{codeblock}
namespace N {
	template<class T1, class T2> class A { };       // primary template
}

using N::A;                     // refers to the primary template

namespace N {
	template<class T> class A<T, T*> { };	// partial specialization
}

A<int,int*> a;			// uses the partial specialization, which is found through
                                // the using declaration which refers to the primary template
\end{codeblock}
\exitexampleb

\pnum
A non-type argument is non-specialized if it is the name of a non-type
parameter.
All other non-type arguments are specialized.

\pnum
Within the argument list of a class template partial specialization,
the following restrictions apply:

\begin{itemize}
\item
A partially specialized non-type argument expression shall not involve
a template parameter of the partial specialization except when the argument
expression is a simple
\grammarterm{identifier}.
\enterexample
\begin{codeblock}
template <int I, int J> struct A {};
template <int I> struct A<I+5, I*2> {}; // error

template <int I, int J> struct B {};
template <int I> struct B<I, I> {};     // OK
\end{codeblock}
\exitexampleb
\item
The type of a template parameter corresponding to a specialized non-type argument
shall not be dependent on a parameter of the specialization.
\enterexample

\begin{codeblock}
template <class T, T t> struct C {};
template <class T> struct C<T, 1>;		// error

template< int X, int (*array_ptr)[X] > class A {};
int array[5];
template< int X > class A<X,&array> { };	// error
\end{codeblock}

\exitexampleb
\item
The argument list of the specialization shall not be identical to the
implicit argument list of the primary template.
\end{itemize}

\pnum
The template parameter list of a specialization shall not contain default
template argument values.\footnote{There is no way in which they could be used.}

\rSec3[temp.class.spec.match]{Matching of class template partial specializations}

\pnum
When a class template is used in a context that requires an instantiation of
the class,
it is necessary to determine whether the instantiation is to be generated
using the primary template or one of the partial specializations.
This is done by matching the template arguments of the class template
specialization with the template argument lists of the partial
specializations.

\begin{itemize}
\item
If exactly one matching specialization is found, the instantiation is
generated from that specialization.
\item
If more than one matching specialization is found,
the partial order rules~(\ref{temp.class.order}) are used to determine
whether one of the specializations is more specialized than the
others.
If none of the specializations is more specialized than all of the
other matching specializations, then the use of the class template
is ambiguous and the program is ill-formed.
\item
If no matches are found, the instantiation is generated from the
primary template.
\end{itemize}

\pnum
A partial specialization matches a given actual template argument
list if the template arguments of the partial specialization can be
deduced from the actual template argument list~(\ref{temp.deduct}).
\enterexample

\begin{codeblock}
A<int, int, 1>   a1;            // uses \#1
A<int, int*, 1>  a2;            // uses \#2, \tcode{T} is \tcode{int}, \tcode{I} is \tcode{1}
A<int, char*, 5> a3;            // uses \#4, \tcode{T} is \tcode{char}
A<int, char*, 1> a4;            // uses \#5, \tcode{T1} is \tcode{int}, \tcode{T2} is \tcode{char}, \tcode{I} is \tcode{1}
A<int*, int*, 2> a5;            // ambiguous: matches \#3 and \#5
\end{codeblock}
\exitexampleb

\pnum
A non-type template argument can also be deduced from the value of an actual
template argument of a non-type parameter of the primary template.
\enterexample
the declaration of
\tcode{a2}
above.
\exitexample

\pnum
In a type name that refers to a class template specialization, (e.g.,
\tcode{A<int, int, 1>})
the argument list must match the template parameter list of the primary
template.
The template arguments of a specialization are deduced from the arguments
of the primary template.

\rSec3[temp.class.order]{Partial ordering of class template specializations}

\pnum
For two class template partial specializations,
the first is at least as specialized as the second if, given the following
rewrite to two function templates, the first function template is at least
as specialized as the second according to the ordering rules for function
templates~(\ref{temp.func.order}):

\begin{itemize}
\item
the first function template has the same template parameters as
the first partial specialization and has a single function parameter
whose type is a class template specialization with the template arguments of
the first partial specialization, and
\item
the second function template has the same template parameters as
the second partial specialization and has a single function parameter
whose type is a class template specialization with the template arguments of
the second partial specialization.
\end{itemize}

\pnum
\enterexample
\begin{codeblock}
template<int I, int J, class T> class X { };
template<int I, int J>          class X<I, J, int> { };		// \#1
template<int I>                 class X<I, I, int> { };		// \#2

template<int I, int J> void f(X<I, J, int>);    // A
template<int I>        void f(X<I, I, int>);    // B
\end{codeblock}

The partial specialization
\tcode{\#2}
is more specialized than the partial specialization
\tcode{\#1}
because the function template
\tcode{B}
is more specialized than the function template
\tcode{A}
according to the ordering rules for function templates.
\exitexample

\rSec3[temp.class.spec.mfunc]{Members of class template specializations}

\pnum
The template parameter list of a member of a class template partial
specialization shall match the template parameter list of the class template
partial specialization.
The template argument list of a member of a class template partial
specialization shall match the template argument list of the class template
partial specialization.
A class template specialization is a distinct template.
The members of the class template partial specialization are
unrelated to the members of the primary template.
Class template partial specialization members that are used in a way that
requires a definition shall be defined; the definitions of members of the
primary template are never used as definitions for members of a class
template partial specialization.
An explicit specialization of a member of a class template partial
specialization is declared in the same way as an explicit specialization of
the primary template.
\enterexample

\begin{codeblock}
// primary template
template<class T, int I> struct A {
	void f();
};

template<class T, int I> void A<T,I>::f() { }

// class template partial specialization
template<class T> struct A<T,2> {
	void f();
	void g();
	void h();
};

// member of class template partial specialization
template<class T> void A<T,2>::g() { }

// explicit specialization
template<> void A<char,2>::h() { }

int main()
{
	A<char,0> a0;
	A<char,2> a2;
	a0.f();                 // OK, uses definition of primary template's member
	a2.g();                 // OK, uses definition of
                                // partial specialization's member
	a2.h();                 // OK, uses definition of
                                // explicit specialization's member
	a2.f();                 // ill-formed, no definition of \tcode{f} for \tcode{A<T,2>}
                                // the primary template is not used here
}
\end{codeblock}
\exitexampleb

\pnum
If a member template of a class template is partially specialized,
the member template partial specializations are member templates of
the enclosing class template;
if the enclosing class template is instantiated~(\ref{temp.inst}, \ref{temp.explicit}),
a declaration for every member template partial specialization is also
instantiated as part of creating the members of the class template
specialization.
If the primary member template is explicitly specialized for a given
(implicit) specialization of the enclosing class template,
the partial specializations of the member template are ignored for this
specialization of the enclosing class template.
If a partial specialization of the member template is explicitly specialized
for a given (implicit) specialization of the enclosing class template,
the primary member template and its other partial specializations are
still considered for this specialization of the enclosing class template.
\enterexample

\begin{codeblock}
template<class T> struct A {
	template<class T2> struct B {};         // \#1
	template<class T2> struct B<T2*> {};    // \#2
};

template<> template<class T2> struct A<short>::B {};	// \#3

A<char>::B<int*>  abcip;	// uses \#2
A<short>::B<int*> absip;	// uses \#3
A<char>::B<int>   abci;		// uses \#1
\end{codeblock}
\exitexampleb

\rSec2[temp.fct]{Function templates}

\pnum
A function template defines an unbounded set of related functions.
\enterexample
a family of sort functions might be declared like this:

\begin{codeblock}
template<class T> class Array { };
template<class T> void sort(Array<T>&);
\end{codeblock}
\exitexampleb

\pnum
A function template can be overloaded with other function templates
and with normal (non-template) functions.
A normal function is not
related to a function template
(i.e., it is never considered to be a specialization),
even if it has the same name and type
as a potentially generated function template specialization.\footnote{That is,
declarations of non-template functions do not merely guide
overload resolution of
function template specializations
with the same name.
If such a non-template function is used in a program, it must be defined;
it will not be implicitly instantiated using the function template definition.}

\rSec3[temp.over.link]{Function template overloading}

\pnum
\indextext{overloading}%
It is possible to overload function templates so that two different
function template specializations have the same type.
\enterexample

\begin{minipage}{.45\hsize}
\begin{codeblock}
// file1.c
template<class T>
    void f(T*);
void g(int* p) {
    f(p); // calls
	  // \tcode{f<int>(int*)}
}
\end{codeblock}
\end{minipage}
\begin{minipage}{.45\hsize}
\begin{codeblock}
// file2.c
template<class T>
    void f(T);
void h(int* p) {
    f(p); // calls
          // \tcode{f<int*>(int*)}
}
\end{codeblock}
\end{minipage}

\exitexampleb

\pnum
Such specializations are distinct functions and do not violate the one
definition rule~(\ref{basic.def.odr}).

\pnum
The signature of a function template specialization consists of the
signature of the function template and of the actual template arguments
(whether explicitly specified or deduced).

\pnum
The signature of a function template consists of its function signature,
its return type and its template parameter list.
The names of the template parameters are significant only for establishing
the relationship between the template parameters and the rest of the
signature.
\enternote
two distinct function templates may have identical function return types and
function parameter lists, even if overload resolution alone cannot distinguish
them.

\begin{codeblock}
template<class T> void f();
template<int I> void f();       // OK: overloads the first template
                                // distinguishable with an explicit template argument list
\end{codeblock}
\exitnoteb

\pnum
When an expression that references a template parameter is used in the
function parameter list or the return type in the declaration of a
function template, the expression that references the template
parameter is part of the signature of the function template.
This is
necessary to permit a declaration of a function template in one
translation unit to be linked with another declaration of the function
template in another translation unit and, conversely, to ensure that
function templates that are intended to be distinct are not linked
with one another.
\enterexample

\begin{codeblock}
template <int I, int J> A<I+J> f(A<I>, A<J>);   // \#1
template <int K, int L> A<K+L> f(A<K>, A<L>);   // same as \#1
template <int I, int J> A<I-J> f(A<I>, A<J>);   // different from \#1
\end{codeblock}
\exitexampleb
\enternote
Most expressions that use template parameters use non-type template
parameters, but it is possible for an expression to reference a type
parameter.
For example, a template type parameter can be used in the
\tcode{sizeof} operator.
\exitnote

\pnum
Two expressions involving template parameters are considered
\term{equivalent}
if two function definitions containing the expressions would satisfy
the one definition rule~(\ref{basic.def.odr}), except that the tokens used
to name the template parameters may differ as long as a token used to
name a template parameter in one expression is replaced by another token
that names the same template parameter in the other expression.
\enterexample

\begin{codeblock}
template <int I, int J> void f(A<I+J>);         // \#1
template <int K, int L> void f(A<K+L>);         // same as \#1
\end{codeblock}
\exitexampleb
Two expressions involving template parameters that are not equivalent are
\term{functionally equivalent}
if, for any given set of template arguments, the evaluation of the
expression results in the same value.

\pnum
Two function templates are
\term{equivalent}
if they are declared in the same scope, have the same name, have
identical template parameter lists, and have return types and parameter
lists that are equivalent using the rules described above to compare
expressions involving
template parameters.
Two function templates are
\term{functionally equivalent}
if they are equivalent except that one or more
expressions
that involve template parameters in the return types and parameter
lists are functionally equivalent using the rules described above to
compare expressions involving
template parameters.
If a program contains declarations of function templates that are
functionally equivalent but not equivalent, the program is ill-formed;
no diagnostic is required.

\pnum
\enternote
This rule guarantees that equivalent declarations will be linked with
one another, while not requiring implementations to use heroic efforts
to guarantee that functionally equivalent declarations will be treated
as distinct.
For example, the last two declarations are functionally
equivalent and would cause a program to be ill-formed:

\begin{codeblock}
// Guaranteed to be the same
template <int I> void f(A<I>, A<I+10>);
template <int I> void f(A<I>, A<I+10>);

// Guaranteed to be different
template <int I> void f(A<I>, A<I+10>);
template <int I> void f(A<I>, A<I+11>);

// Ill-formed, no diagnostic required
template <int I> void f(A<I>, A<I+10>);
template <int I> void f(A<I>, A<I+1+2+3+4>);
\end{codeblock}
\exitnoteb

\rSec3[temp.func.order]{Partial ordering of function templates}

\pnum
\indextext{overloading!resolution!template}%
\indextext{ordering!function template partial}%
If a function template is overloaded,
the use of a function template specialization might be ambiguous because
template argument deduction~(\ref{temp.deduct}) may associate the function
template specialization with more than one function template declaration.
\term{Partial ordering}
of overloaded function template declarations is used in the following contexts
to select the function template to which a function template specialization
refers:

\begin{itemize}
\item
during overload resolution for a call to a function template specialization~(\ref{over.match.best});
\item
when the address of a function template specialization is taken;
\item
when a placement operator delete that is a
function template
specialization
is selected to match a placement operator new~(\ref{basic.stc.dynamic.deallocation}, \ref{expr.new});
\item
when a friend function declaration~(\ref{temp.friend}), an
explicit instantiation~(\ref{temp.explicit}) or an explicit specialization~(\ref{temp.expl.spec}) refers to
a function template specialization.
\end{itemize}

\pnum
Given two overloaded function templates, whether one is more specialized
than another can be determined by transforming each template in turn
and using argument deduction~(\ref{temp.deduct}) to compare it to the
other.

\pnum
The transformation used is:

\begin{itemize}
\item
For each type template parameter, synthesize a unique type and substitute
that for each occurrence of that parameter in the function parameter list,
or for a template conversion function, in the return type.
\item
For each non-type template parameter, synthesize a unique value of the
appropriate type and substitute that for each occurrence of that parameter
in the function parameter list, or for a template conversion function,
in the return type.
\item
For each template template parameter, synthesize a unique class template
and substitute that for each occurrence of that parameter in the function
parameter list, or for a template conversion function, in the return type.
\end{itemize}

\pnum
Using the transformed function parameter list, perform argument deduction
against the other function template. The transformed template is at least
as specialized as the other if, and only if, the deduction succeeds and
the deduced parameter types are an exact match (so the deduction does not
rely on implicit conversions).

\pnum
A template is more specialized than another if, and only if, it is at least
as specialized as the other template and that template is not at least
as specialized as the first.
\enterexample

\begin{codeblock}
template<class T> struct A { A(); };

template<class T> void f(T);
template<class T> void f(T*);
template<class T> void f(const T*);

template<class T> void g(T);
template<class T> void g(T&);

template<class T> void h(const T&);
template<class T> void h(A<T>&);

void m() {
	const int *p;
	f(p);			// \tcode{f(const T*)} is more specialized than \tcode{f(T)} or \tcode{f(T*)}
	float x;
	g(x);			// Ambiguous: \tcode{g(T)} or \tcode{g(T\&)}
	A<int> z;
	h(z);			// overload resolution selects \tcode{h(A<T>\&)}
	const A<int> z2;
	h(z2);			// \tcode{h(const T\&)} is called because \tcode{h(A<T>\&)} is not callable
}
\end{codeblock}
\exitexampleb

\pnum
The presence of unused ellipsis and default arguments has no effect on
the partial ordering of function templates.
\enterexample

\begin{codeblock}
template<class T> void f(T);            // \#1
template<class T> void f(T*, int=1);    // \#2
template<class T> void g(T);            // \#3
template<class T> void g(T*, ...);      // \#4

\end{codeblock}
\begin{codeblock}
int main() {
	int* ip;
	f(ip);			// calls \#2
	g(ip);			// calls \#4
}
\end{codeblock}
\exitexampleb

\rSec1[temp.res]{Name resolution}

\pnum
\indextext{overloading!resolution!template name}%
\indextext{lookup!template name}%
Three kinds of names can be used within a template definition:

\begin{itemize}
\item
The name of the template itself,
and names declared within the template itself.
\item
Names dependent on a
\grammarterm{template-parameter}~(\ref{temp.dep}).
\item
Names from scopes which are visible within the template definition.
\end{itemize}

\pnum
A name used in a template declaration or definition and that is
dependent on a
\grammarterm{template-parameter}
is assumed not to name a type unless
the applicable name lookup finds a type name or the name
is qualified by the keyword
\tcode{typename}.
\enterexample

\begin{codeblock}
// no \tcode{B} declared here

class X;

template<class T> class Y {
	class Z;                // forward declaration of member class

	void f() {
		X* a1;          // declare pointer to \tcode{X}
		T* a2;          // declare pointer to \tcode{T}
		Y* a3;          // declare pointer to \tcode{Y<T>}
		Z* a4;          // declare pointer to \tcode{Z}
		typedef typename T::A TA;
		TA* a5;			// declare pointer to \tcode{T}'s \tcode{A}
		typename T::A* a6;      // declare pointer to \tcode{T}'s \tcode{A}
		T::A* a7;               // \tcode{T::A} is not a type name:
					// multiply \tcode{T::A} by \tcode{a7}; ill-formed,
					// no visible declaration of \tcode{a7}
		B* a8;                  // \tcode{B} is not a type name:
					// multiply \tcode{B} by \tcode{a8}; ill-formed,
					// no visible declarations of \tcode{B} and \tcode{a8}
	}
};
\end{codeblock}
\exitexampleb

\pnum
A \nonterminal{qualified-id} that refers to a type and in which the
\nonterminal{nested-name-specifier} depends on a
\nonterminal{typename-parameter}~(\ref{temp.dep}) shall be prefixed by
the keyword \tcode{typename} to indicate that the
\grammarterm{qualified-id} denotes a type, forming an
\grammarterm{elaborated-type-specifier}~(\ref{dcl.type.elab}).

\begin{ncbnf}
\nontermdef{elaborated-type-specifier}\br
  \terminal{. . .}\br
  \terminal{typename} \terminal{::}\opt nested-name-specifier identifier\br
  \terminal{typename} \terminal{::}\opt nested-name-specifier \terminal{template\opt} template-id\br
  \terminal{. . .}
\end{ncbnf}

\pnum
If a specialization of a template is instantiated for a set of
\grammarterm{template-argument}{s}
such that the
\grammarterm{qualified-id}
prefixed by
\tcode{typename}
does not denote a type, the specialization is ill-formed.
The usual qualified name lookup~(\ref{basic.lookup.qual}) is used to find the
\grammarterm{qualified-id}
even in the presence of
\tcode{typename}.
\enterexample

\begin{codeblock}
struct A {
	struct X { };
	int X;
};
template<class T> void f(T t) {
	typename T::X x;	// ill-formed: finds the data member \tcode{X}
				// not the member type \tcode{X}
}
\end{codeblock}
\exitexampleb

\pnum
The keyword \tcode{typename} shall only be used in template declarations
and definitions, including in the return type of a function template or
member function template, in the return type for the definition of a member
function of a class template or of a class nested within a class template,
and in the \grammarterm{type-specifier} for the definition of a static
member of a class template or of a class nested within a class template.
The keyword \tcode{typename} shall be applied only to qualified names, but
those names need not be dependent. The keyword \tcode{typename} shall be
used only in contexts in which dependent names can be used. This includes
template declarations and definitions but excludes explicit specialization
declarations and explicit instantiation declarations. The keyword
\tcode{typename} is not permitted in a \grammarterm{base-specifier} or in a
\grammarterm{mem-initializer}; in these contexts a \grammarterm{qualified-id}
that depends on a \grammarterm{template-parameter}~(\ref{temp.dep}) is
implicitly assumed to be a type name.

\pnum
Within the definition of a class template or within the definition of a
member of a class template, the keyword \tcode{typename}
is not required when referring to the unqualified name of a previously
declared member of the class template that declares a type.
The keyword \tcode{typename} shall always be specified when the member
is referred to using a qualified name, even if the qualifier is simply
the class template name.
\enterexample

\begin{codeblock}
template<class T> struct A {
    typedef int B;
    A::B b;			// ill-formed: typename required before \tcode{A::B}
    void f(A<T>::B);		// ill-formed: typename required before \tcode{A<T>::B}
    typename A::B g();		// OK
};
\end{codeblock}

The keyword \tcode{typename} is required whether the qualified name is
\tcode{A} or \tcode{A<T>} because \tcode{A} or \tcode{A<T>} are synonyms
within a class template with the parameter list \tcode{<T>}.
\exitexample

\pnum
\indextext{checking!syntax}%
\indextext{checking!point~of error}%
Knowing which names are type names allows the syntax of every template
definition to be checked.
No diagnostic shall be issued for a template
definition for which a valid specialization can be generated.
If no valid specialization can be generated
for a template definition,
and that template is not instantiated, the template definition is ill-formed,
no diagnostic required.
If a type used in a non-dependent name is incomplete
at the point at which a template is defined but is
complete at the point at which an instantiation is done,
and if the completeness of that type affects whether or
not the program is well-formed or affects the semantics
of the program, the program is ill-formed; no diagnostic
is required.
\enternote
if a template is instantiated, errors will be diagnosed according
to the other rules in this Standard.
Exactly when these errors are diagnosed is a quality of implementation issue.
\exitnote
\enterexample

\begin{codeblock}
int j;
template<class T> class X {
	// ...
	void f(T t, int i, char* p)
	{
		t = i;		// diagnosed if \tcode{X::f} is instantiated
				// and the assignment to \tcode{t} is an error
		p = i;		// may be diagnosed even if \tcode{X::f} is
				// not instantiated
		p = j;		// may be diagnosed even if \tcode{X::f} is
				// not instantiated
	}
	void g(T t) {
		+;		// may be diagnosed even if \tcode{X::g} is
				// not instantiated
	}
};
\end{codeblock}
\exitexampleb

\pnum
When looking for the declaration of a name used in a template definition,
the usual lookup rules~(\ref{basic.lookup.unqual}, \ref{basic.lookup.koenig})
are used for nondependent names.
The lookup of names dependent on the template parameters
is postponed until the actual template argument is known~(\ref{temp.dep}).
\enterexample

\begin{codeblock}
#include <iostream>
using namespace std;

template<class T> class Set {
	T* p;
	int cnt;
public:
	Set();
	Set<T>(const Set<T>&);
	void printall()
	{
		for (int i = 0; i<cnt; i++)
			cout << p[i] << '@\textbackslash@n';
	}
	// ...
};
\end{codeblock}

in the example,
\tcode{i}
is the local variable
\tcode{i}
declared in
\tcode{printall},
\tcode{cnt}
is the member
\tcode{cnt}
declared in
\tcode{Set},
and
\tcode{cout}
is the standard output stream declared in
\tcode{iostream}.
However, not every declaration can be found this way; the resolution of
some names must be postponed
until the actual
\grammarterm{template-argument}{s}
are known.
For example, even though the name
\tcode{operator\shl}
is known within the definition of
\tcode{printall()}
and a declaration of it can be found in
\tcode{<iostream>},
the actual declaration of
\tcode{operator\shl}
needed to print
\tcode{p[i]}
cannot be known until it is known what type
\tcode{T}
is~(\ref{temp.dep}).
\exitexample

\pnum
If a name does not depend on a
\grammarterm{template-parameter}
(as defined in~\ref{temp.dep}), a declaration (or set of declarations) for that
name shall be in scope at the point where the name appears in the template
definition; the name is bound to the declaration (or declarations) found
at that point and this binding is not affected by declarations that are
visible at the point of instantiation.
\enterexample

\begin{codeblock}
void f(char);

template<class T> void g(T t)
{
	f(1);			// \tcode{f(char)}
	f(T(1));		// dependent
	f(t);			// dependent
	dd++;			// not dependent
				// error: declaration for dd not found
}

void f(int);

double dd;
void h()
{
	g(2);			// will cause one call of \tcode{f(char)} followed
				// by two calls of \tcode{f(int)}
	g('a');			// will cause three calls of \tcode{f(char)}
}
\end{codeblock}
\exitexampleb

\pnum
\enternote
for purposes of name lookup,
default arguments of function templates and default arguments of
member functions of class templates are considered definitions~(\ref{temp.decls}).
\exitnoteb

\rSec2[temp.local]{Locally declared names}

\pnum
Like normal (non-template) classes, class templates have an
injected-class-name (clause~\ref{class}).
The
injected-class-name can be used with or without a
\grammarterm{template-argument-list}.
When it is used without a
\grammarterm{template-argument-list},
it is equivalent to the injected-class-name
followed by the \grammarterm{template-parameter}{s} of the class template
enclosed in \tcode{<>}.
When it is used with a
\grammarterm{template-argument-list},
it refers to the specified class template specialization, which could be
the current specialization or another specialization.

\pnum
Within the scope of a class template specialization or
partial specialization, when the injected-class-name is
not followed by a \tcode{<}, it is equivalent to the injected-class-name
followed by the
\grammarterm{template-argument}{s}
of the class template specialization or partial
specialization enclosed in
\tcode{<>}.
\enterexample
\begin{codeblock}
template<class T> class Y;
template<> class Y<int> {
    Y* p;                       // meaning \tcode{Y<int>}
    Y<char>* q;                 // meaning \tcode{Y<char>}
};
\end{codeblock}
\exitexampleb

\pnumalt
The injected-class-name of a class template or class
template specialization can be used either with or without a
\grammarterm{template-argument-list}
wherever it is in scope.
\enterexample

\begin{codeblock}
template <class T> struct Base {
    Base* p;
};

template <class T> struct Derived: public Base<T> {
    typename Derived::Base* p;  // meaning \tcode{Derived::Base<T>}
};
\end{codeblock}
\exitexampleb

\pnumalt
A lookup that finds an injected-class-name~(\ref{class.member.lookup}) can result in an ambiguity in
certain cases (for example, if it is found in more than one
base class).
If all of the injected-class-names that are
found refer to specializations of the same class template,
and if the name is followed by a
\grammarterm{template-argument-list},
the reference refers to the class template itself and not a
specialization thereof, and is not ambiguous.
\enterexample

\begin{codeblock}
template <class T> struct Base { };
template <class T> struct Derived: Base<int>, Base<char> {
    typename Derived::Base b;           // error: ambiguous
    typename Derived::Base<double> d;   // OK
};
\end{codeblock}
\exitexampleb

\pnumalt
When the normal name of the template (i.e., the name from
the enclosing scope, not the injected-class-name) is
used without a \grammarterm{template-argument-list},
it refers to the class template itself and not a
specialization of the template.
\enterexample

\begin{codeblock}
template<class T> class X {
    X* p;			// meaning \tcode{X<T>}
    X<T>* p2;
    X<int>* p3;
    ::X* p4;			// error: missing template argument list
				// \tcode{::X} does not refer to the injected-class-name
};
\end{codeblock}
\exitexampleb

\pnumalt
Within the scope of a class template, when the unqualified name of a nested
class of the class template is referred to, it is equivalent to the name of
the nested class qualified by the name of the enclosing class template.
\enterexample

\begin{codeblock}
template <class T> struct A {
    class B { };		// \tcode{B} is equivalent to \tcode{A::B}, which is equivalent to \tcode{A<T>::B},

				// which is dependent.
    class C : B { };
}
\end{codeblock}

\exitexampleb

\pnum
The scope of a \grammarterm{template-parameter} extends from its point of
declaration until the end of its template. A \grammarterm{template-parameter}
hides any entity with the same name in the enclosing scope.
\enternote this implies that a \grammarterm{template-parameter} can be used
in the declaration of subsequent \grammarterm{template-parameter}{s} and
their default arguments but cannot be used in preceding
\grammarterm{template-parameter}{s} or their default arguments.
For example,

\begin{codeblock}
template<class T, T* p, class U = T> class X { /* ... */ };
template<class T> void f(T* p = new T);
\end{codeblock}

This also implies that a \grammarterm{template-parameter} can be used in
the specification of base classes. For example,

\begin{codeblock}
template<class T> class X : public Array<T> { /* ... */ };
template<class T> class Y : public T { /* ... */ };
\end{codeblock}

The use of a \grammarterm{template-parameter} as a base class implies that
a class used as a \grammarterm{template-argument} must be defined and
not just declared when the class template is instantiated.
\exitnote

\pnum
A
\grammarterm{template-parameter}
shall not be redeclared within its scope (including nested scopes).
A
\grammarterm{template-parameter}
shall not have the same name as the template name.
\enterexample

\begin{codeblock}
template<class T, int i> class Y {
	int T;			// error: template-parameter redeclared
	void f() {
		char T;		// error: template-parameter redeclared
	}
};

template<class X> class X;	// error: template-parameter redeclared
\end{codeblock}
\exitexampleb

\pnum
In the definition of a member of
a class template that appears outside of the class template definition,
the name of a member of this template hides the name of a
\grammarterm{template-parameter}.
\enterexample

\begin{codeblock}
template<class T> struct A {
	struct B @\texttt{\{ /* ... */ \};}@
	void f();
};

template<class B> void A<B>::f() {
	B b;			// \tcode{A}'s \tcode{B}, not the template parameter
}
\end{codeblock}
\exitexampleb

\pnum
In the definition of a member of a class template that appears outside of the
namespace containing the class template definition,
the name of a
\grammarterm{template-parameter}
hides the name of a member of this namespace.
\enterexample

\begin{codeblock}
namespace N {
	class C { };
	template<class T> class B {
		void f(T);
	};
}
template<class C> void N::B<C>::f(C) {
	C b;			// \tcode{C} is the template parameter, not \tcode{N::C}
}
\end{codeblock}
\exitexampleb

\pnum
In the definition of a class template or in the definition of a member of such
a template that appears outside of the template definition,
for each base class which does not depend on a
\grammarterm{template-parameter}~(\ref{temp.dep}), if the name of the base class
or the name of a member of the
base class is the same as the name of a
\grammarterm{template-parameter},
the base class name or member name hides the
\grammarterm{template-parameter}
name~(\ref{basic.scope.hiding}).
\enterexample

\begin{codeblock}
struct A {
	struct B { /* ... */ };
	int a;
	int Y;
};

template<class B, class a> struct X : A {
	B b;			// \tcode{A}'s \tcode{B}
	a b;			// error: \tcode{A}'s \tcode{a} isn't a type name
};
\end{codeblock}
\exitexampleb

\rSec2[temp.dep]{Dependent names}

\pnum
\indextext{name!dependent}%
Inside a template, some constructs have semantics which may differ from one
instantiation to another.
Such a construct
\term{depends}
on the template parameters.
In particular, types and expressions may depend on the type
and/or
value of
template parameters (as determined by the template arguments) and this determines
the context for name lookup for certain names.
Expressions may be
\grammarterm{type-dependent}
(on the type of a template parameter) or
\grammarterm{value-dependent}
(on the value of a non-type template parameter).
In an expression of the form:

\begin{ncbnftab}
postfix-expression \terminal{(} expression-list\opt \terminal{)}
\end{ncbnftab}

where the
\grammarterm{postfix-expression}
is an
\grammarterm{identifier},
the
\grammarterm{identifier}
denotes a
\indextext{name!dependent}%
\term{dependent name}
if and only if any of the expressions in the
\grammarterm{expression-list}
is a type-dependent expression~(\ref{temp.dep.expr}).
If an operand of an operator is a type-dependent expression, the operator
also denotes a dependent name.
Such names are unbound and
are looked up at the point of the template instantiation~(\ref{temp.point}) in
both the context of the template definition and the
context of the point of instantiation.

\pnum
\enterexample
\begin{codeblock}
template<class T> struct X : B<T> {
	typename T::A* pa;
	void f(B<T>* pb) {
		static int i = B<T>::i;
		pb->j++;
	}
};
\end{codeblock}

the base class name
\tcode{B<T>},
the type name
\tcode{T::A},
the names
\tcode{B<T>::i}
and
\tcode{pb->j}
explicitly depend on the
\grammarterm{template-parameter}.
\exitexampleb

\pnum
In the definition of a class template or a member of a class template,
if a base class of the class template depends on a
\grammarterm{template-parameter},
the base class scope is not examined during unqualified
name lookup either at the point of definition of the
class template or member or during an instantiation of
the class template or member.
\enterexample

\begin{codeblock}
typedef double A;
template<class T> class B {
	typedef int A;
};
template<class T> struct X : B<T> {
	A a;			// \tcode{a} has type \tcode{double}
};
\end{codeblock}

The type name
\tcode{A}
in the definition of
\tcode{X<T>}
binds to the typedef name defined in the global
namespace scope, not to the typedef name
defined in the base class
\tcode{B<T>}.
\exitexample
\enterexample

\begin{codeblock}
struct A {
	struct B { /* ... */ };
	int a;
	int Y;
};

int a;

template<class T> struct Y : T {
	struct B { /* ... */ };
	B b;				// The \tcode{B} defined in \tcode{Y}
	void f(int i) { a = i; }	// \tcode{::a}
	Y* p;				// \tcode{Y<T>}
};

Y<A> ya;
\end{codeblock}

The members
\tcode{A::B},
\tcode{A::a},
and
\tcode{A::Y}
of the template argument
\tcode{A}
do not affect the binding of names in
\tcode{Y<A>}.
\exitexample

\rSec3[temp.dep.type]{Dependent types}

\pnum
A type is dependent if it is
\begin{itemize}
\item
a template parameter,
\item
a \grammarterm{qualified-id} with a \grammarterm{nested-name-specifier}
which contains a \grammarterm{class-name} that names a dependent type or
whose \grammarterm{unqualified-id} names a dependent type,
\item
a cv-qualified type where the cv-unqualified type is dependent,
\item
a compound type constructed from any dependent type,
\item
an array type constructed from any dependent type or whose
size is specified by a constant expression that is value-dependent,
\item
a
\grammarterm{template-id}
in which either the template name is a template parameter or any of the
template arguments is a dependent type or an expression that is type-dependent
or value-dependent.
\end{itemize}

\rSec3[temp.dep.expr]{Type-dependent expressions}

\pnum
Except as described below, an expression is type-dependent if any
subexpression is type-dependent.

\pnum
\tcode{this}
is type-dependent if the class type of the enclosing member function is
dependent~(\ref{temp.dep.type}).

\pnum
An
\grammarterm{id-expression}
is type-dependent if it contains

\begin{itemize}
\item
an
\grammarterm{identifier}
that was declared with a dependent type,

\item
a
\grammarterm{template-id}
that is dependent,

\item
a
\grammarterm{conversion-function-id}
that specifies a dependent type,

\item
a
\grammarterm{nested-name-specifier}
that contains a
\grammarterm{class-name}
that names a dependent type.
\end{itemize}

Expressions of the following forms are type-dependent only if the type
specified by the
\grammarterm{type-id},
\grammarterm{simple-type-specifier}
or
\grammarterm{new-type-id}
is dependent, even if any subexpression is type-dependent:

\begin{ncbnftab}
simple-type-specifier \terminal{(} expression-list\opt \terminal{)}\br
\terminal{::\opt new} new-placement\opt new-type-id new-initializer\opt\br
\terminal{::\opt new} new-placement\opt \terminal{(} type-id \terminal{)} new-initializer\opt\br
\terminal{dynamic_cast <} type-id \terminal{> (} expression \terminal{)}\br
\terminal{static_cast <} type-id \terminal{> (} expression \terminal{)}\br
\terminal{const_cast <} type-id \terminal{> (} expression \terminal{)}\br
\terminal{reinterpret_cast <} type-id \terminal{> (} expression \terminal{)}\br
\terminal{(} type-id \terminal{)} cast-expression
\end{ncbnftab}

\pnum
Expressions of the following forms are never type-dependent (because the type
of the expression cannot be dependent):

\begin{ncbnftab}
literal\br
postfix-expression \terminal{.} pseudo-destructor-name\br
postfix-expression \terminal{->} pseudo-destructor-name\br
\terminal{sizeof} unary-expression\br
\terminal{sizeof (} type-id \terminal{)}\br
\terminal{typeid (} expression \terminal{)}\br
\terminal{typeid (} type-id \terminal{)}\br
\terminal{::\opt delete} cast-expression\br
\terminal{::\opt delete [ ]} cast-expression\br
\terminal{throw} assignment-expression\opt
\end{ncbnftab}

\rSec3[temp.dep.constexpr]{Value-dependent expressions}

\pnum
Except as described below, a constant expression is value-dependent if any
subexpression is value-dependent.

\pnum
An
\grammarterm{identifier}
is value-dependent if it is:

\begin{itemize}
\item
a name declared with a dependent type,
\item
the name of a non-type template parameter,
\item
a constant with integral or enumeration type and is initialized with an
expression that is value-dependent.
\end{itemize}

Expressions of the following form are value-dependent if the
\grammarterm{unary-expression}
is type-dependent or the
\grammarterm{type-id}
is dependent (even if \tcode{sizeof} \grammarterm{unary-expression} and
\tcode{sizeof (} \grammarterm{type-id} \tcode{)} are not type-dependent):

\begin{ncbnftab}
\terminal{sizeof} unary-expression\br
\terminal{sizeof (} type-id \terminal{)}
\end{ncbnftab}

\pnum
Expressions of the following form are value-dependent if either the
\grammarterm{type-id}
or
\grammarterm{simple-type-specifier}
is dependent or the
\grammarterm{expression}
or
\grammarterm{cast-expression}
is value-dependent:

\begin{ncbnftab}
simple-type-specifier \terminal{(} expression-list\opt \terminal{)}\br
\terminal{static_cast <} type-id \terminal{> (} expression \terminal{)}\br
\terminal{const_cast <} type-id \terminal{> (} expression \terminal{)}\br
\terminal{reinterpret_cast <} type-id \terminal{> (} expression \terminal{)}\br
\terminal{(} type-id \terminal{)} cast-expression
\end{ncbnftab}

\rSec3[temp.dep.temp]{Dependent template arguments}

\pnum
A type
\grammarterm{template-argument}
is dependent if the type it specifies is dependent.

\pnum
An integral non-type
\grammarterm{template-argument}
is dependent if the constant expression it specifies is value-dependent.

\pnum
A non-integral non-type
\grammarterm{template-argument}
is dependent if its type is dependent or it has either of the following
forms

\begin{ncbnftab}
qualified-id\br
\terminal{\&} qualified-id
\end{ncbnftab}

and contains a \grammarterm{nested-name-specifier} which specifies a
\grammarterm{class-name} that names a dependent type.

\pnum
A template
\grammarterm{template-argument}
is dependent if it names a
\grammarterm{template-parameter}
or is a
\grammarterm{qualified-id}
with a
\grammarterm{nested-name-specifier}
which contains a
\grammarterm{class-name}
that names a dependent type.

\rSec2[temp.nondep]{Non-dependent names}

\pnum
Non-dependent names used in a template definition are found using the
usual name lookup and bound at the point they are used.
\enterexample

\begin{codeblock}
void g(double);
void h();

template<class T> class Z {
public:
	void f() {
		g(1);		// calls \tcode{g(double)}
		h++;		// ill-formed: cannot increment function;
				// this could be diagnosed either here or
				// at the point of instantiation
	}
};

void g(int);			// not in scope at the point of the template
				// definition, not considered for the call \tcode{g(1)}
\end{codeblock}
\exitexampleb

\rSec2[temp.dep.res]{Dependent name resolution}

\pnum
\indextext{name!dependent}%
In resolving dependent names, names from the following sources are considered:

\begin{itemize}
\item
Declarations that are visible at the point of definition of the
template.
\item
Declarations from namespaces associated with the types of the
function arguments both from the instantiation context~(\ref{temp.point})
and from the definition context.
\end{itemize}

\rSec3[temp.point]{Point of instantiation}

\pnum
\indextext{instantiation!point~of}%
For a function template specialization, a member function template
specialization, or a specialization for a member function or static data member
of a class template,
if the specialization is implicitly instantiated because it is referenced
from within another template specialization and
the context from which it is referenced depends on a template parameter,
the point of instantiation of the specialization is the point of instantiation
of the enclosing specialization.
Otherwise, the point of instantiation for such a specialization immediately
follows the namespace scope declaration
or definition that refers to the specialization.

\pnum
If a function template or member function of a class template is called
in a way which uses the definition of a default argument of that function
template or member function,
the point of instantiation of the default argument is the point of
instantiation of the function template or member function specialization.

\pnum
For a class template specialization, a class member template specialization,
or a specialization for a class member of a class template,
if the specialization is implicitly instantiated because it is referenced
from within another template specialization,
if the context from which the specialization is referenced depends on a
template parameter,
and if the specialization is not instantiated previous to the instantiation of
the enclosing template,
the point of instantiation is immediately before the point of instantiation of
the enclosing template.
Otherwise, the point of instantiation for such a specialization immediately
precedes the namespace scope declaration
or definition that refers to the specialization.

\pnum
If a virtual function is implicitly instantiated, its point of instantiation
is immediately following the point of instantiation of its enclosing class
template specialization.

\pnum
An explicit instantiation definition is an instantiation
point for the specialization or specializations specified by the explicit
instantiation directive.

\pnum
The instantiation context of an expression that depends on the template
arguments is the set of declarations with external linkage declared prior to the
point of instantiation of the template specialization in the same translation
unit.

\pnum
A specialization for a function template, a member function template,
or of a member function or static data member of a class template may have
multiple points of instantiations within a translation unit.
A specialization for a class template has at most one point of instantiation
within a translation unit.
A specialization for any template may have points of instantiation in multiple
translation units.
If two different points of instantiation give a template specialization
different meanings according to the one definition rule~(\ref{basic.def.odr}),
the program is ill-formed, no diagnostic required.

\rSec3[temp.dep.candidate]{Candidate functions}

\pnum
\indextext{functions!candidate}%
For a function call that depends on a template parameter,
if the function name is an \grammarterm{unqualified-id} but not a
\grammarterm{template-id}, the candidate functions are found using the
usual lookup rules~(\ref{basic.lookup.unqual}, \ref{basic.lookup.koenig})
except that:

\begin{itemize}
\item
For the part of the lookup using unqualified name lookup~(\ref{basic.lookup.unqual}),
only function declarations with external linkage
from the template definition context are found.
\item
For the part of the lookup using associated namespaces~(\ref{basic.lookup.koenig}),
only function declarations with external linkage found in either the template
definition context or the template instantiation context are found.
\end{itemize}

If the call would be ill-formed or would find a better match had the lookup
within the associated namespaces considered all the function declarations with
external linkage introduced in those namespaces in all translation units,
not just considering those declarations found in the template definition and
template instantiation contexts, then the program has undefined behavior.

\rSec2[temp.inject]{Friend names declared within a class template}

\pnum
Friend classes or functions can be declared within a class template.
When a template is instantiated, the names of its friends are treated
as if the specialization had been explicitly declared at its point of
instantiation.

\pnum
As with non-template classes, the names of namespace-scope friend
functions of a class template specialization are not visible during
an ordinary lookup unless explicitly declared at namespace scope~(\ref{class.friend}).
Such names may be found under the rules for associated
classes~(\ref{basic.lookup.koenig}).\footnote{Friend declarations do not
introduce new names into any scope, either
when the template is declared or when it is instantiated.}
\enterexample
\begin{codeblock}
template<typename T> class number {
public:
	number(int);
	// ...
	friend number gcd(number x, number y);
	// ...
};

void g()
{
	number<double> a(3), b(4);
	// ...
	a = gcd(a,b);		// finds \tcode{gcd} because \tcode{number<double>} is an
				// associated class, making \tcode{gcd} visible
				// in its namespace (global scope)
	b = gcd(3,4);		// ill-formed; \tcode{gcd} is not visible
}
\end{codeblock}
\exitexampleb

\rSec1[temp.spec]{Template instantiation and specialization}

\pnum
\indextext{specialization!template}%
The act of instantiating a function, a class, a member of a class template or
a member template is referred to as
\term{template instantiation}.

\pnum
A function instantiated from a function template is called an instantiated
function.
A class instantiated from a class template is called an instantiated class.
A member function, a member class, or a static data member of a class template
instantiated from the member definition of the class template is called,
respectively, an instantiated member function, member class or static data
member.
A member function instantiated from a member function template is called an
instantiated member function.
A member class instantiated from a member class template is called an
instantiated member class.

\pnum
An explicit specialization may be declared for a function template,
a class template, a member of a class template or a member template.
An explicit specialization declaration is introduced by
\tcode{template<>}.
In an explicit specialization declaration for a class template,
a member of a class template or a class member template,
the name of the class that is explicitly specialized shall be a
\grammarterm{template-id}.
In the explicit specialization declaration for a function template or
a member function template,
the name of the function or member function explicitly specialized may be a
\grammarterm{template-id}.
\enterexample

\begin{codeblock}
template<class T = int> struct A {
	static int x;
};
template<class U> void g(U) { }

template<> struct A<double> { };        // specialize for \tcode{T == double}
template<> struct A<> { };              // specialize for \tcode{T == int}
template<> void g(char) { }             // specialize for \tcode{U == char}
                                        // \tcode{U} is deduced from the parameter type
template<> void g<int>(int) { }         // specialize for \tcode{U == int}
template<> int A<char>::x = 0;          // specialize for \tcode{T == char}

template<class T = int> struct B {
	static int x;
};
template<> int B<>::x = 1;              // specialize for \tcode{T == int}
\end{codeblock}
\exitexampleb

\pnum
An instantiated template specialization can be either implicitly
instantiated~(\ref{temp.inst}) for a given argument list or be explicitly
instantiated~(\ref{temp.explicit}).
A specialization is a class, function, or class member that is either
instantiated or explicitly specialized~(\ref{temp.expl.spec}).

\pnum
No program shall explicitly instantiate any template more than once,
both explicitly instantiate and explicitly specialize a template, or
specialize a template more than once for a given set of
\grammarterm{template-argument}{s}. An implementation is not required
to diagnose a violation of this rule.

\pnum
Each class template specialization instantiated from a template has its own
copy of any static members.
\enterexample

\begin{codeblock}
template<class T> class X {
	static T s;
	// ...
};
template<class T> T X<T>::s = 0;
X<int> aa;
X<char*> bb;
\end{codeblock}

\tcode{X<int>}
has a static member
\tcode{s}
of type
\tcode{int}
and
\tcode{X<char*>}
has a static member
\tcode{s}
of type
\tcode{char*}.
\exitexample

\rSec2[temp.inst]{Implicit instantiation}

\pnum
\indextext{instantiation!template implicit}%
Unless a class template specialization has been explicitly
instantiated~(\ref{temp.explicit}) or explicitly
specialized~(\ref{temp.expl.spec}),
the class template specialization is implicitly instantiated when the
specialization is referenced in a context that requires a completely-defined
object type or when the completeness of the class type affects the semantics
of the program.
The implicit instantiation of a class template specialization causes
the implicit instantiation of the declarations, but not of the definitions or
default arguments, of the class member functions,
member classes, static data members and member templates; and it causes the
implicit instantiation of the definitions of member anonymous unions.
Unless a member of a class template or a member template has been explicitly
instantiated or explicitly specialized,
the specialization of the member is implicitly instantiated when the
specialization is referenced in a context that requires the member definition
to exist;
in particular, the initialization (and any associated side-effects) of a
static data member does not occur unless the static data member is itself used
in a way that requires the definition of the static data member to exist.

\pnum
Unless a function template specialization has been explicitly instantiated or
explicitly specialized,
the function template specialization is implicitly instantiated when the
specialization is referenced in a context that requires a function definition
to exist.
Unless a call is to a function template explicit specialization or
to a member function of an explicitly specialized class template,
a default argument for a function template or a member function of a
class template is implicitly instantiated when the function is
called in a context that requires the value of the default argument.

\pnum
\enterexample
\begin{codeblock}
template<class T> class Z {
public:
	void f();
	void g();
};

void h()
{
	Z<int> a;		// instantiation of class \tcode{Z<int>} required
	Z<char>* p;		// instantiation of class \tcode{Z<char>} not
				// required
	Z<double>* q;		// instantiation of class \tcode{Z<double>}
				// not required

	a.f();			// instantiation of \tcode{Z<int>::f()} required
	p->g();			// instantiation of class \tcode{Z<char>} required, and
				// instantiation of \tcode{Z<char>::g()} required
}
\end{codeblock}

Nothing in this example requires
\tcode{class}
\tcode{Z<double>},
\tcode{Z<int>::g()},
or
\tcode{Z<char>::f()}
to be implicitly instantiated.
\exitexample

\pnum
A class template specialization is implicitly instantiated if the
class type is used in a context that requires a completely-defined
object type or if the completeness of the class type affects the
semantics of the program;
in particular, if an expression whose type is a class template
specialization is involved in overloaded resolution, pointer conversion,
pointer to member conversion, the class template specialization is
implicitly instantiated~(\ref{basic.def.odr});
in addition, a class template specialization is implicitly instantiated
if the operand of a delete expression is of class type or is of pointer
to class type and the class type is a template specialization.
\enterexample

\begin{codeblock}
template<class T> class B { /* ... */ };
template<class T> class D : public B<T> { /* ... */ };

void f(void*);
void f(B<int>*);

void g(D<int>* p, D<char>* pp, D<double> ppp)
{
	f(p);			// instantiation of \tcode{D<int>} required: call \tcode{f(B<int>*)}

	B<char>* q = pp;	// instantiation of \tcode{D<char>} required:
				// convert \tcode{D<char>*} to \tcode{B<char>*}

	delete ppp;		// instantiation of \tcode{D<double>} required
}
\end{codeblock}
\exitexampleb

\pnum
If the overload resolution process can determine the correct function to
call without instantiating a class template definition, it is unspecified
whether that instantiation actually takes place.
\enterexample

\begin{codeblock}
template <class T> struct S {
	operator int();
};

void f(int);
void f(S<int>&);
void f(S<float>);

void g(S<int>& sr) {
	f(sr);			// instantiation of \tcode{S<int>} allowed but not required
				// instantiation of \tcode{S<float>} allowed but not required
};
\end{codeblock}
\exitexampleb

\pnum
If an implicit instantiation of a class template specialization is required and
the template is declared but not defined, the program is ill-formed.
\enterexample

\begin{codeblock}
template<class T> class X;

X<char> ch;			// error: definition of \tcode{X} required
\end{codeblock}
\exitexampleb

\pnum
The implicit instantiation of a class template does not cause any static data
members of that class to be implicitly instantiated.

\pnum
If a function template or a member function template specialization is used in
a way that involves overload resolution,
a declaration of the specialization is implicitly instantiated~(\ref{temp.over}).

\pnum
An implementation shall not implicitly instantiate a function template,
a member template, a non-virtual member function, a member class or a
static data member of a class template that does not require instantiation.
It is unspecified whether or not an implementation implicitly instantiates a
virtual member function of a class template if the virtual member function would
not otherwise be instantiated.
The use of a template specialization in a default argument
shall not cause the template to be implicitly instantiated except that a
class template may be instantiated where its complete type is needed to determine
the correctness of the default argument.
The use of a default argument in a
function call causes specializations in the default argument to be implicitly
instantiated.

\pnum
Implicitly instantiated class and function template specializations are placed
in the namespace where the template is defined.
Implicitly instantiated specializations for members of a class template are
placed in the namespace where the enclosing class template is defined.
Implicitly instantiated member templates are placed in the namespace where the
enclosing class or class template is defined.
\enterexample

\begin{codeblock}
namespace N {
	template<class T> class List {
	public:
		T* get();
	// ...
	};
}

template<class K, class V> class Map {
	N::List<V> lt;
	V get(K);
	// ...
};

void g(Map<char*,int>& m)
{
	int i = m.get("Nicholas");
	// ...
}
\end{codeblock}

a call of
\tcode{lt.get()}
from
\tcode{Map<char*,int>::get()}
would place
\tcode{List<int>::get()}
in the namespace
\tcode{N}
rather than in the global namespace.
\exitexample

\pnum
If a function template
\tcode{f}
is called in a way that requires a default argument expression to be used,
the dependent names are looked up, the semantics constraints are checked,
and the instantiation of any template used in the default argument expression
is done as if the default argument expression had been
an expression used in a function template specialization with the same scope,
the same template parameters and the same access as that of the function template
\tcode{f}
used at that point.
This analysis is called
\term{default argument instantiation}.
The instantiated default argument is then used as the argument of
\tcode{f}.

\pnum
Each default argument is instantiated independently.
\enterexample

\begin{codeblock}
template<class T> void f(T x, T y = ydef(T()), T z = zdef(T()));

class  A { };

A zdef(A);

void g(A a, A b, A c) {
	f(a, b, c);		// no default argument instantiation
	f(a, b);		// default argument \tcode{z = zdef(T())} instantiated
	f(a);			// ill-formed; \tcode{ydef} is not declared
}
\end{codeblock}
\exitexampleb

\pnum
\enternote
\ref{temp.point} defines the point of instantiation of a template specialization.
\exitnote

\pnum
There is an implementation-defined quantity that specifies the limit on
the total depth of recursive instantiations, which could involve more than one
template.
The result of an infinite recursion in instantiation is undefined.
\enterexample

\begin{codeblock}
template<class T> class X {
	X<T>* p;		// OK
	X<T*> a;		// implicit generation of \tcode{X<T>} requires
				// the implicit instantiation of \tcode{X<T*>} which requires
				// the implicit instantiation of \tcode{X<T**>} which ...
};
\end{codeblock}
\exitexampleb

\rSec2[temp.explicit]{Explicit instantiation}

\pnum
\indextext{instantiation!explicit}%
A class, a function or member template specialization can be explicitly
instantiated from its template.
A member function, member class or static data member of a class template can
be explicitly instantiated from the member definition associated with its class
template.

\pnum
The syntax for explicit instantiation is:

\begin{bnf}
\nontermdef{explicit-instantiation}\br
  \terminal{template} declaration
\end{bnf}

If the explicit instantiation is for a class,
a function or a member template specialization,
the
\grammarterm{unqualified-id}
in the
\grammarterm{declaration}
shall be either a
\grammarterm{template-id}
or, where all template arguments can be deduced, a
\grammarterm{template-name}.
\enternote
the declaration may declare a
\grammarterm{qualified-id},
in which case the
\grammarterm{unqualified-id}
of the
\grammarterm{qualified-id}
must be a
\grammarterm{template-id}.
\exitnote
If the explicit instantiation is for a member function, a member class or
a static data member of a class template specialization,
the name of the class template specialization in the
\grammarterm{qualified-id}
for the member \grammarterm{declarator} shall be a \grammarterm{template-id}.
\enterexample

\begin{codeblock}
template<class T> class Array { void mf(); };
template class Array<char>;
template void Array<int>::mf();

template<class T> void sort(Array<T>& v) { /* ... */ }
template void sort(Array<char>&);       // argument is deduced here

namespace N {
	template<class T> void f(T&) { }
}
template void N::f<int>(int&);
\end{codeblock}
\exitexampleb

\pnum
A declaration of a function template shall be in scope at the point of the
explicit instantiation of the function template. A definition of the class
or class template containing a member function template shall be in scope at
the point of the explicit instantiation of the member function template.
A definition of a class template or class member template shall be in scope
at the point of the explicit instantiation of the class template or class
member template. A definition of a class template shall be in scope at the
point of an explicit instantiation of a member function or a static data member
of the class template. A definition of a member class of a class template
shall be in scope at the point of an explicit instantiation of the member
class.  If the \grammarterm{declaration}
of the explicit instantiation names an implicitly-declared special member
function (clause~\ref{special}), the program is ill-formed.

\pnum
The definition of a non-exported function template, a non-exported member
function template, or a non-exported member function or static
data member of a class template shall be present in every
translation unit in which it is explicitly instantiated.

\pnum
An explicit instantiation of a class or function template specialization is
placed in the namespace in which the template is defined.
An explicit instantiation for a member of a class template is placed in
the namespace where the enclosing class template is defined.
An explicit instantiation for a member template is placed in the namespace
where the enclosing class or class template is defined.
\enterexample

\begin{codeblock}
namespace N {
	template<class T> class Y { void mf() { } };
}

template class Y<int>;          // error: class template \tcode{Y} not visible
                                // in the global namespace

using N::Y;
template class Y<int>;          // OK: explicit instantiation in namespace \tcode{N}

template class N::Y<char*>;             // OK: explicit instantiation in namespace \tcode{N}
template void N::Y<double>::mf();       // OK: explicit instantiation
                                        // in namespace \tcode{N}
\end{codeblock}
\exitexampleb

\pnum
A trailing
\grammarterm{template-argument}
can be left unspecified in an explicit instantiation of a function template
specialization or of a member function template specialization provided
it can be deduced from the type of a function parameter~(\ref{temp.deduct}).
\enterexample

\begin{codeblock}
template<class T> class Array { /* ... */ };
template<class T> void sort(Array<T>& v);

// instantiate \tcode{sort(Array<int>\&)} -- template-argument deduced
template void sort<>(Array<int>&);
\end{codeblock}
\exitexampleb

\pnum
The explicit instantiation of a class template specialization implies the
instantiation of all of its members not previously explicitly specialized in
the translation unit containing the explicit instantiation.

\pnum
The usual access checking rules do not apply to names used to specify
explicit instantiations.
\enternote
In particular, the template arguments and names used in the function
declarator (including parameter types, return types and exception
specifications) may be private types or objects which would normally
not be accessible and the template may be a member template or member
function which would not normally be accessible.
\exitnote

\pnum
An explicit instantiation does not constitute a use of a default argument,
so default argument instantiation is not done.
\enterexample

\begin{codeblock}
char* p = 0;
template<class T> T g(T = &p);
template int g<int>(int);       // OK even though \tcode{\&p} isn't an \tcode{int}.
\end{codeblock}
\exitexampleb

\rSec2[temp.expl.spec]{Explicit specialization}

\pnum
\indextext{specialization!template explicit}%
An explicit specialization of any of the following:

\begin{itemize}
\item
function template
\item
class template
\item
member function of a class template
\item
static data member of a class template
\item
member class of a class template
\item
member class template of a class template
\item
member function template of a class template
\end{itemize}

can be declared by a declaration introduced by
\tcode{template<>};
that is:
\indextext{\idxgram{explicit-specialization}}%

\begin{bnf}
\nontermdef{explicit-specialization}\br
  \terminal{template < >} declaration
\end{bnf}

\enterexample
\begin{codeblock}
template<class T> class stream;

template<> class stream<char> { /* ... */ };

template<class T> class Array { /* ... */ };
template<class T> void sort(Array<T>& v) { /* ... */ }

template<> void sort<char*>(Array<char*>&) ;
\end{codeblock}

Given these declarations,
\tcode{stream<char>}
will be used as the definition of streams of
\tcode{char}s;
other streams will be handled by class template specializations instantiated
from the class template.
Similarly,
\tcode{sort<char*>}
will be used as the sort function for arguments
of type
\tcode{Array<char*>};
other
\tcode{Array}
types will be sorted by functions generated from the template.
\exitexample

\pnum
An explicit specialization shall be declared in the namespace of which the
template is a member, or, for member templates, in the namespace of which
the enclosing class or enclosing class template is a member.
An explicit specialization of a member function, member class or static
data member of a class template shall be declared in the namespace of which
the class template is a member. Such a declaration may also be a definition.
If the declaration is not a definition, the specialization may be defined
later in the namespace in which the explicit specialization was declared,
or in a namespace that encloses the one in which the explicit specialization
was declared.

\pnum
A declaration of a function template or class template being explicitly
specialized shall be in scope at the point of declaration of an explicit
specialization.
\enternote
a declaration, but not a definition of the template is required.
\exitnote
The definition of a class or class template shall be in scope at the point of
declaration of an explicit specialization for a member template of the class
or class template.
\enterexample

\begin{codeblock}
template<> class X<int> { /* ... */ };          // error: \tcode{X} not a template

template<class T> class X;

template<> class X<char*> { /* ... */ };        // OK: \tcode{X} is a template
\end{codeblock}
\exitexampleb

\pnum
A member function, a member class or a static data member of a class template
may be explicitly specialized for a class specialization that is implicitly
instantiated;
in this case, the definition of the class template shall be in scope at the
point of declaration of the explicit specialization for the member of the class
template.
If such an explicit specialization for the member of a class template names an
implicitly-declared special member function (clause~\ref{special}),
the program is ill-formed.

\pnum
A member of an explicitly specialized class is not implicitly
instantiated from the member declaration of the class template;
instead, the member of the class template specialization shall itself be
explicitly defined.
In this case, the definition of the class template explicit specialization
shall be in scope at the point of declaration of the explicit specialization
of the member.
The definition of an explicitly specialized class is unrelated to the
definition of a generated specialization.
That is, its members need
not have the same names, types, etc. as the members of the a generated
specialization.
Definitions of members of an explicitly specialized
class are defined in the same manner as members of normal classes, and
not using the explicit specialization syntax.
\enterexample

\begin{codeblock}
template<class T> struct A {
	void f(T) { /* ... */ }
};

template<> struct A<int> {
	void f(int);
};

void h()
{
	A<int> a;
	a.f(16);		// \tcode{A<int>::f} must be defined somewhere
}

// explicit specialization syntax not used for a member of
// explicitly specialized class template specialization
void A<int>::f() { /* ... */ }
\end{codeblock}
\exitexampleb

\pnum
If a template, a member template or the member of a class template is explicitly
specialized then that specialization shall be declared before the first use of
that specialization that would cause an implicit instantiation to take place,
in every translation unit in which such a use occurs;
no diagnostic is required.
If the program does not provide a definition for an explicit specialization and
either the specialization is used in a way that would cause an implicit
instantiation to take place or the member is a virtual member function,
the program is ill-formed, no diagnostic required.
An implicit instantiation is never generated for an explicit specialization
that is declared but not defined.
\enterexample

\begin{codeblock}
template<class T> class Array { /* ... */ };
template<class T> void sort(Array<T>& v) { /* ... */ }

void f(Array<String>& v)
{
	sort(v);		// use primary template
				// \tcode{sort(Array<T>\&)}, \tcode{T} is \tcode{String}
}

template<> void sort<String>(Array<String>& v); // error: specialization
                                                // after use of primary template
template<> void sort<>(Array<char*>& v);        // OK: \tcode{sort<char*>} not yet used
\end{codeblock}
\exitexampleb

\pnum
The placement of explicit specialization declarations for function templates, class
templates, member functions of class templates, static data members of class
templates, member classes of class templates, member class templates of class
templates, member function templates of class templates, member functions of
member templates of class templates, member functions of member templates of
non-template classes, member function templates of member classes of class
templates, etc., and the placement of partial specialization declarations
of class templates, member class templates of non-template classes, member
class templates of class templates, etc., can affect whether a program is
well-formed according to the relative positioning of the explicit specialization
declarations and their points of instantiation in the translation unit as
specified above and below.
\indextext{immolation!self}%
When writing a specialization, be careful about its location;
or to make it compile will be such a trial as to kindle its self-immolation.

\pnum
When a specialization for which an explicit specialization exists is used
within the instantiation of an exported template, and the unspecialized
template name is non-dependent in the exported template, a declaration
of the explicit specialization shall be declared before the definition of
the exported template, in the translation unit containing that definition.
\enterexample

\begin{codeblock}
// file \#1
#include <vector>
// Primary class template \tcode{vector}
export template<class T> void f(t) {
	std::vector<T>; vec;	// should match the specialization
	/* ... */
}

// file \#2
#include <vector>
class B { };
// Explicit specialization of \tcode{vector} for \tcode{vector<B>}
namespace std {
	template<> class vector<B> { /* ... */ };
}
template<class T> void f(T);
void g(B b) {
	f(b);			// ill-formed:
				// \tcode{f<B>} should refer to \tcode{vector<B>}, but the
				// specialization was not declared with the
				// definition of \tcode{f} in file \#1
}
\end{codeblock}
\exitexampleb

\pnum
A template explicit specialization is in the scope of the namespace in which
the template was defined.
\enterexample

\begin{codeblock}
namespace N {
	template<class T> class X { /* ... */ };
	template<class T> class Y { /* ... */ };

	template<> class X<int> { /* ... */ };    // OK: specialization
                                                // in same namespace
	template<> class Y<double>;             // forward declare intent to
                                                // specialize for \tcode{double}
}

template<> class N::Y<double> { /* ... */ };      // OK: specialization
                                                // in same namespace
\end{codeblock}
\exitexampleb

\pnum
A
\grammarterm{template-id}
that names a class template explicit specialization that has been declared but
not defined can be used exactly like the names of other incompletely-defined
classes~(\ref{basic.types}).
\enterexample

\begin{codeblock}
template<class T> class X;      // \tcode{X} is a class template
template<> class X<int>;

X<int>* p;                      // OK: pointer to declared class \tcode{X<int>}
X<int> x;                       // error: object of incomplete class \tcode{X<int>}
\end{codeblock}
\exitexampleb

\pnum
A trailing
\grammarterm{template-argument}
can be left unspecified in the
\grammarterm{template-id}
naming an explicit function template specialization
provided it can be deduced from the function argument type.
\enterexample

\begin{codeblock}
template<class T> class Array { /* ... */ };
template<class T> void sort(Array<T>& v);

// explicit specialization for \tcode{sort(Array<int>\&)}
// with deduced template-argument of type \tcode{int}
template<> void sort(Array<int>&);
\end{codeblock}
\exitexampleb

\pnum
\enternote This paragraph is intentionally empty. \exitnote

\pnum
A function with the same name as a template and a type that exactly matches that
of a template specialization is not an explicit specialization~(\ref{temp.fct}).

\pnum
An explicit specialization of a function template is inline
only if it is explicitly declared to be, and independently of whether its
function template is.
\enterexample

\begin{codeblock}
template<class T> void f(T) { /* ... */ }
template<class T> inline T g(T) { /* ... */ }

template<> inline void f<>(int) { /* ... */ }   // OK: inline
template<> int g<>(int) { /* ... */ }           // OK: not inline
\end{codeblock}
\exitexampleb

\pnum
An explicit specialization of a static data member of a template is a
definition if the declaration includes an initializer;
otherwise, it is a declaration.
\enternote
there is no syntax for the definition of a static data member of a template
that requires default initialization.

\begin{codeblock}
template<> X Q<int>::x;
\end{codeblock}

This is a declaration regardless of whether X can be default
initialized~(\ref{dcl.init}).
\exitnote

\pnum
A member or a member template of a class template may be explicitly specialized
for a given implicit instantiation of the class template, even if the member
or member template is defined in the class template definition.
An explicit specialization of a member or member template is specified using the
template specialization syntax.
\enterexample

\begin{codeblock}
template<class T> struct A {
	void f(T);
	template<class X1> void g1(T, X1);
	template<class X2> void g2(T, X2);
	void h(T) { }
};

// specialization
template<> void A<int>::f(int);

// out of class member template definition
template<class T> template<class X1> void A<T>::g1(T, X1) { }

// member template specialization
template<> template<class X1> void A<int>::g1(int, X1);

// member template specialization
template<> template<>
	void A<int>::g1(int, char);     // \tcode{X1} deduced as \tcode{char}
template<> template<>
	void A<int>::g2<char>(int, char); // \tcode{X2} specified as \tcode{char}

// member specialization even if defined in class definition
template<> void A<int>::h(int) { }
\end{codeblock}
\exitexampleb

\pnum
A member or a member template may be nested within many enclosing class
templates.
If the declaration of an explicit specialization for such a member
appears in namespace scope, the member declaration shall be preceded by a
\tcode{template<>}
for each enclosing class template that is explicitly specialized.
\enterexample

\begin{codeblock}
template<class T1> class A {
	template<class T2> class B {
		void mf();
	};
};
template<> template<> class A<int>::B<double>;
template<> template<> void A<char>::B<char>::mf();
\end{codeblock}
\exitexampleb

\pnum
In an explicit specialization declaration for a member of a class template or
a member template that appears in namespace scope,
the member template and some of its enclosing class templates may remain
unspecialized,
except that the declaration shall not explicitly specialize a class member
template if its enclosing class templates are not explicitly specialized
as well.
In such explicit specialization declaration, the keyword
\tcode{template}
followed by a
\grammarterm{template-parameter-list}
shall be provided instead of the
\tcode{template<>}
preceding the explicit specialization declaration of the member.
The types of the
\grammarterm{template-parameters}
in the
\grammarterm{template-parameter-list}
shall be the same as those specified in the primary template definition.
\enterexample

\begin{codeblock}
template<class T1> class A {
	template<class T2> class B {
		template<class T3> void mf1(T3);
		void mf2();
	};
};
template<> template<class X>
  class A<int>::B { };
template<> template<> template<class T>
  void A<int>::B<double>::mf1(T t) { }
template<class Y> template<>
  void A<Y>::B<double>::mf2() { }   // ill-formed; \tcode{B<double>} is specialized but
                                    // its enclosing class template \tcode{A} is not
\end{codeblock}
\exitexampleb

\pnum
A specialization of a member function template or member class template of
a non-specialized class template is itself a template.

\pnum
An explicit specialization declaration shall not be a friend declaration.

\pnum
Default function arguments shall not be specified in a declaration or
a definition for one of the following explicit specializations:

\begin{itemize}
\item
the explicit specialization of a function template;
\item
the explicit specialization of a member function template;
\item
the explicit specialization of a member function of a class template where
the class template specialization to which the member function specialization
belongs is implicitly instantiated.
\enternote
default function arguments may be specified in the declaration or
definition of a member function of a class template specialization that is
explicitly specialized.
\exitnote
\end{itemize}

\rSec1[temp.fct.spec]{Function template specializations}

\pnum
\indextext{template!function}%
A function instantiated from a function template is called a function template
specialization; so is an explicit specialization of a function template.
Template arguments can either be explicitly specified when naming the function
template specialization or be deduced~(\ref{temp.deduct}) from the context,
e.g. from the function arguments in a call to the function template
specialization.

\pnum
Each function template specialization instantiated from a template
has its own copy of any static variable.
\enterexample

\begin{codeblock}
template<class T> void f(T* p)
{
	static T s;
	// ...
};

void g(int a, char* b)
{
	f(&a);			// calls \tcode{f<int>(int*)}
	f(&b);			// calls \tcode{f<char*>(char**)}
}
\end{codeblock}

Here
\tcode{f<int>(int*)}
has a static variable
\tcode{s}
of type
\tcode{int}
and
\tcode{f<char*>(char**)}
has a static variable
\tcode{s}
of type
\tcode{char*}.
\exitexample

\rSec2[temp.arg.explicit]{Explicit template argument specification}

\pnum
\indextext{specification!template argument}%
Template arguments can be specified when referring to a function
template specialization by qualifying the function template
name with the list of
\grammarterm{template-argument}{s}
in the same way as
\grammarterm{template-argument}{s}
are specified in uses of a class template specialization.
\enterexample

\begin{codeblock}
template<class T> void sort(Array<T>& v);
void f(Array<dcomplex>& cv, Array<int>& ci)
{
	sort<dcomplex>(cv);     // \tcode{sort(Array<dcomplex>\&)}
	sort<int>(ci);          // \tcode{sort(Array<int>\&)}
}
\end{codeblock}

and

\begin{codeblock}
template<class U, class V> U convert(V v);

void g(double d)
{
	int i = convert<int,double>(d);		// \tcode{int convert(double)}
	char c = convert<char,double>(d);	// \tcode{char convert(double)}
}
\end{codeblock}
\exitexampleb

\pnum
A template argument list may be specified when referring to a specialization
of a function template

\begin{itemize}
\item
when a function is called,
\item
when the address of a function is taken, when a function initializes a
reference to function, or when a pointer to member function is formed,
\item
in an explicit specialization,
\item
in an explicit instantiation, or
\item
in a friend declaration.
\end{itemize}

Trailing template arguments that can be deduced~(\ref{temp.deduct})
may be omitted from the list of explicit
\grammarterm{template-argument}{s}.
If all of the template arguments can be deduced, they may all be omitted;
in this case, the empty template argument list
\tcode{<>}
itself may also be omitted.
\enterexample

\begin{codeblock}
template<class X, class Y> X f(Y);
void g()
{
	int i = f<int>(5.6);    // \tcode{Y} is deduced to be \tcode{double}
	int j = f(5.6);         // ill-formed: \tcode{X} cannot be deduced
}
\end{codeblock}
\exitexampleb
\enternote
An empty template argument list can be used to indicate that a given
use refers to a specialization of a function template even when a
normal (i.e., nontemplate) function is visible that would otherwise be used.
For example:

\begin{codeblock}
template <class T> int f(T);    // \#1
int f(int);                     // \#2
int k = f(1);                   // uses \#2
int l = f<>(1);                 // uses \#1
\end{codeblock}
\exitnoteb

\pnum
Template arguments that are present shall be specified in the declaration
order of their corresponding
\grammarterm{template-parameter}{s}.
The template argument list shall not specify more
\grammarterm{template-argument}{s}
than there are corresponding
\grammarterm{template-parameter}{s}
\enterexample

\begin{codeblock}
template<class X, class Y, class Z> X f(Y,Z);
void g()
{
	f<int,char*,double>("aa",3.0);
	f<int,char*>("aa",3.0);	// \tcode{Z} is deduced to be \tcode{double}
	f<int>("aa",3.0);	// \tcode{Y} is deduced to be \tcode{char*}, and
				// \tcode{Z} is deduced to be \tcode{double}
	f("aa",3.0);            // error: \tcode{X} cannot be deduced
}
\end{codeblock}
\exitexampleb

\pnum
Implicit conversions (clause~\ref{conv}) will be performed on a function argument
to convert it to the type of the corresponding function parameter if
the parameter type contains no
\grammarterm{template-parameter}{s}
that participate in template argument deduction.
\enternote
template parameters do not participate in template argument deduction if
they are explicitly specified.
For example,

\begin{codeblock}
template<class T> void f(T);

class Complex {
	// ...
	Complex(double);
};

void g()
{
	f<Complex>(1);		// OK, means \tcode{f<Complex>(Complex(1))}
}
\end{codeblock}
\exitnoteb

\pnum
\enternote
because the explicit template argument list follows the function
template name, and because conversion member function templates and
constructor member function templates are called without using a
function name, there is no way to provide an explicit template
argument list for these function templates.
\exitnote
% L7048 USA Core3 1.18 / 14.8.1 [temp.arg.explicit]
% WG21 decided not to address this issue except to document that
% argument-dependent lookup does not apply in this context.

\pnum
\enternote
For simple function names, argument dependent lookup~(\ref{basic.lookup.koenig})
applies even when the function name is not visible within the scope of the call.
This is because the call still has the syntactic form of a function call~(\ref{basic.lookup.unqual}).
But when a function template with explicit template arguments is used,
the call does not have the correct syntactic form unless there is a function
template with that name visible at the point of the call.
If no such name is visible,
the call is not syntactically well-formed and argument-dependent lookup
does not apply.
If some such name is visible,
argument dependent lookup applies and additional function templates
may be found in other namespaces.
\enterexample

% Argument added to f per Usenet posting from martin von Loewis, 6 Sep 1998
\begin{codeblock}
namespace A {
	struct B { };
	template<int X> void f(B);
}
namespace C {
	template<class T> void f(T t);
}
void g(A::B b) {
	f<3>(b);		// ill-formed: not a function call
	A::f<3>(b);		// well-formed
	C::f<3>(b);		// ill-formed; argument dependent lookup
				// applies only to unqualified names
	using C::f;
	f<3>(b);		// well-formed because \tcode{C::f} is visible; then
				// \tcode{A::f} is found by argument dependent lookup
}
\end{codeblock}
\exitexampleb
\exitnoteb

\rSec2[temp.deduct]{Template argument deduction}

\pnum
When a function template specialization is referenced, all of the
template arguments must have values. The values can be
either explicitly specified or, in some cases, deduced from the use.
\enterexample

\begin{codeblock}
void f(Array<dcomplex>& cv, Array<int>& ci)
{
	sort(cv);		// calls \tcode{sort(Array<dcomplex>\&)}
	sort(ci);		// calls \tcode{sort(Array<int>\&)}
}
\end{codeblock}

and

\begin{codeblock}
void g(double d)
{
	int i = convert<int>(d);	// calls \tcode{convert<int,double>(double)}
	int c = convert<char>(d);	// calls \tcode{convert<char,double>(double)}
}
\end{codeblock}
\exitexampleb

\pnum
When an explicit template argument list is specified, the template
arguments must be compatible with the template parameter list and must
result in a valid function type as described below; otherwise type
deduction fails.  Specifically, the following steps are performed when
evaluating an explicitly specified template argument list with respect
to a given function template:

\begin{itemize}
\item The specified template arguments must match the template parameters in
kind (i.e., type, nontype, template), and there must not be more arguments
than there are parameters; otherwise type deduction fails.

\item Nontype arguments must match the types of the corresponding nontype
template parameters, or must be convertible to the types of the
corresponding nontype parameters as specified in~\ref{temp.arg.nontype}, otherwise type deduction fails.

\item All references in the function type of the function template to the
corresponding template parameters are replaced by the specified template
argument values. If a substitution in a template parameter or in the
function type of the function template results in an invalid type, type
deduction fails.
\enternote The equivalent substitution in exception specifications is down
only when the function is instantiated, at which point a program is ill-formed
if the substitution results in an invalid type.\exitnote
Type deduction may fail for the following reasons:

\begin{itemize}
\item Attempting to create an array with an element type that is \tcode{void},
a function type, or a reference type, or attempting to create an array with
a size that is zero or negative.
\enterexample
\begin{codeblock}
template <class T> int f(T[5]);
int I = f<int>(0);
int j = f<void>(0);		// invalid array
\end{codeblock}
\exitexample

\item Attempting to use a type that is not a class type in a qualified name.
\enterexample
\begin{codeblock}
template <class T> int f(typename T::B*);
int i = f<int>(0);
\end{codeblock}
\exitexample

\item Attempting to use a type in the qualifier portion of a qualified name
that names a type when that type does not contain the specified member, or
if the specified member is not a type where a type is required.
\enterexample
\begin{codeblock}
template <class T> int f(typename T::B*);
struct A {};
struct C { int B; };
int i = f<A>(0);
int j = f<C>(0);
\end{codeblock}
\exitexample

\item Attempting to create a pointer to reference type.

\item Attempting to create a reference to a reference type or a reference
to \tcode{void}.

\item Attempting to create ``pointer to member of \tcode{T}'' when \tcode{T}
is not a class type.
\enterexample
\begin{codeblock}
template <class T> int f(int T::*);
int i = f<int>(0);
\end{codeblock}
\exitexample

\item Attempting to perform an invalid conversion in either a template
argument expression, or an expression used in the function declaration.
\enterexample
\begin{codeblock}
template <class T, T*> int f(int);
int i2 = f<int,1>(0);		// can't conv 1 to \tcode{int*}
\end{codeblock}
\exitexample

\item Attempting to create a function type in which a parameter has a type
of \tcode{void}.

\item Attempting to create a \term{cv-qualified} function type.

\end{itemize}

\end{itemize}

\pnum
After this substitution is performed, the function parameter type
adjustments described in~\ref{dcl.fct} are performed.
\enterexample
A parameter type of ``\tcode{void ()(const int, int[5])}'' becomes
``\tcode{void(*)(int,int*)}''.
\exitexample
\enternote
A top-level qualifier in a function parameter declaration does not affect
the function type but still affects the type of the function parameter
variable within the function.
\exitnoteb
\enterexample

\begin{codeblock}
template <class T> void f(T t);
template <class X> void g(const X x);
template <class Z> void h(Z, Z*);

int main()
{
	// \#1: function type is \tcode{f(int)}, \tcode{t} is non \tcode{const}
	f<int>(1);

	// \#2: function type is \tcode{f(int)}, \tcode{t} is \tcode{const}
	f<const int>(1);

	// \#3: function type is \tcode{g(int)}, \tcode{x} is \tcode{const}
	g<int>(1);

	// \#4: function type is \tcode{g(int)}, \tcode{x} is \tcode{const}
	g<const int>(1);

	// \#5: function type is \tcode{h(int, const int*)}
	h<const int>(1,0);
}
\end{codeblock}
\exitexampleb
\enternote
\tcode{f<int>(1)} and \tcode{f<const int>(1)} call distinct functions
even though both of the functions called have the same function type.
\exitnoteb

\pnum
The resulting substituted and adjusted function type is used as
the type of the function template for template argument deduction.
When all template arguments have been deduced, all uses of template
parameters in nondeduced contexts are replaced with the corresponding deduced
argument values. If the substitution results in an
invalid type, as described above, type deduction fails.

\pnum
Except as described above, the use of an invalid value shall not cause
type deduction to fail.
\enterexample
In the following example 1000 is converted to \tcode{signed char} and results
in an implementation-defined value as specified in~(\ref{conv.integral}).
In other words, both templates are considered even though 1000,
when converted to \tcode{signed char}, results in an implementation-defined
value.

\begin{codeblock}
template <int> int f(int);
template <signed char> int f(int);
int i1 = f<1>(0);               // ambiguous
int i2 = f<1000>(0);            // ambiguous
\end{codeblock}
\exitexampleb

\rSec3[temp.deduct.call]{Deducing template arguments from a function call}

\pnum
Template argument deduction is done by comparing each function
template parameter type (call it
\tcode{P})
with the type of the corresponding argument of the call (call it
\tcode{A})
as described below.

\pnum
If
\tcode{P}
is not a reference type:

\begin{itemize}
\item
If
\tcode{A}
is an array type, the pointer type produced by the array-to-pointer
standard conversion~(\ref{conv.array}) is used in place of
\tcode{A}
for type deduction;
otherwise,
\item
If
\tcode{A}
is a function type, the pointer type produced by the
function-to-pointer standard conversion~(\ref{conv.func}) is used in place
of
\tcode{A}
for type
deduction; otherwise,
\item
If
\tcode{A}
is a cv-qualified type, the top level cv-qualifiers of
\tcode{A}'s
type are ignored for type deduction.
\end{itemize}

If
\tcode{P}
is a cv-qualified type, the top level cv-qualifiers of
\tcode{P}'s
type are ignored for type deduction.
If
\tcode{P}
is a reference type, the type
referred to by
\tcode{P}
is used for type deduction.

\pnum
In general, the deduction process attempts to find template argument
values that will make the deduced
\tcode{A}
identical to
\tcode{A}
(after
the type
\tcode{A}
is transformed as described above).
However, there are
three cases that allow a difference:

\begin{itemize}
\item
If the original
\tcode{P}
is a reference type, the deduced
\tcode{A}
(i.e.,
the type referred to by the reference) can be more cv-qualified than
\tcode{A}.
\item
\tcode{A}
can be another pointer or pointer to member type that can be converted
to the deduced
\tcode{A}
via a qualification conversion~(\ref{conv.qual}).

\item
If
\tcode{P}
is a class, and
\tcode{P}
has the form
\grammarterm{template-id},
then
\tcode{A}
can be a derived class of the deduced
\tcode{A}.
Likewise, if \tcode{P} is a pointer to a class of the form
\grammarterm{template-id},
\tcode{A}
can be a pointer to a derived class pointed to by the deduced
\tcode{A}.
\end{itemize}

These alternatives are considered only if type deduction would
otherwise fail.
If they yield more than one possible deduced
\tcode{A},
the type deduction fails.
\enternote
if a
\grammarterm{template-parameter}
is not used in any of the function parameters of a function template,
or is used only in a non-deduced context, its corresponding
\grammarterm{template-argument}
cannot be deduced from a function call and the
\grammarterm{template-argument}
must be explicitly specified.
\exitnote

\rSec3[temp.deduct.funcaddr]{Deducing template arguments taking the address of a function template}

\pnum
Template arguments can be deduced from the type specified when taking
the address of an overloaded function~(\ref{over.over}).
The function template's function type and the specified type
are used as the types of
\tcode{P}
and
\tcode{A},
and the deduction is done as
described in~\ref{temp.deduct.type}.

\rSec3[temp.deduct.conv]{Deducing conversion function template arguments}

\pnum
Template argument deduction is done by comparing the return type of
the template conversion function (call it \tcode{P}) with the type that is
required as the result of the conversion (call it \tcode{A})
as described in~\ref{temp.deduct.type}.

\pnum
If
\tcode{A}
is not a reference type:

\begin{itemize}
\item
If
\tcode{P}
is an array type, the pointer type produced by the
array-to-pointer standard conversion~(\ref{conv.array}) is used in place of
\tcode{P}
for type
deduction; otherwise,
\item
If
\tcode{P}
is a function type, the pointer type produced by the
function-to-pointer standard conversion~(\ref{conv.func}) is used in place of
\tcode{P}
for
type deduction; otherwise,
\item
If
\tcode{P}
is a cv-qualified type, the top level cv-qualifiers of
\tcode{P}'s
type are ignored for type deduction.
\end{itemize}

If
\tcode{A}
is a cv-qualified type, the top level cv-qualifiers of
\tcode{A}'s
type are ignored for type deduction.
If
\tcode{A}
is a reference type, the type referred to by
\tcode{A}
is used for type deduction.

\pnum
In general, the deduction process attempts to find template argument
values that will make the deduced
\tcode{A}
identical to
\tcode{A}.
However, there are two cases that allow a difference:

\begin{itemize}
\item
If the original
\tcode{A}
is a reference type,
\tcode{A}
can be more cv-qualified
than the deduced
\tcode{A}
(i.e., the type referred to by the reference)
\item
The deduced
\tcode{A}
can be another pointer or pointer to member type that
can be converted to
\tcode{A}
via a qualification conversion.
\end{itemize}

These alternatives are considered only if type deduction would
otherwise fail.
If they yield more than one possible deduced
\tcode{A},
the type deduction fails.

\rSec3[temp.deduct.type]{Deducing template arguments from a type}

\pnum
Template arguments can be deduced in several different contexts, but
in each case a type that is specified in terms of template parameters
(call it
\tcode{P})
is compared with an actual type (call it
\tcode{A}),
and an attempt is made to find template argument values (a type for a type
parameter, a value for a non-type parameter, or a template for a
template parameter) that will make
\tcode{P},
after substitution of the deduced values (call it the deduced
\tcode{A}),
compatible with
\tcode{A}.

\pnum
In some cases, the deduction is done using a single set of types
\tcode{P}
and
\tcode{A},
in other cases, there will be a set of corresponding types
\tcode{P}
and
\tcode{A}.
Type deduction is done
independently for each
\tcode{P/A}
pair, and the deduced template
argument values are then combined.
If type deduction cannot be done
for any
\tcode{P/A}
pair, or if for any pair the deduction leads to more than
one possible set of deduced values, or if different pairs yield
different deduced values, or if any template argument remains neither
deduced nor explicitly specified, template argument deduction fails.

\pnum
A given type
\tcode{P}
can be composed from a number of other
types, templates, and non-type values:

\begin{itemize}
\item
A function type includes the types of each of the function parameters
and the return type.
\item
A pointer to member type includes the type of the class object pointed to
and the type of the member pointed to.
\item
A type that is a specialization of a class template (e.g.,
\tcode{A<int>})
includes the types, templates, and non-type values referenced by the
template argument list of the specialization.
\item
An array type includes the array element type and the value of the
array bound.
\end{itemize}

In most cases, the types, templates, and non-type values that are used
to compose
\tcode{P}
participate in template argument deduction.
That is,
they may be used to determine the value of a template argument, and
the value so determined must be consistent with the values determined
elsewhere.
In certain contexts, however, the value does not
participate in type deduction, but instead uses the values of template
arguments that were either deduced elsewhere or explicitly specified.
If a template parameter is used only in nondeduced contexts and is not
explicitly specified, template argument deduction fails.

\pnum
The nondeduced contexts are:

\indextext{context!non-deduced}%
\begin{itemize}
\item
The
\grammarterm{nested-name-specifier}
of a type that was specified using a
\grammarterm{qualified-id}.
\item
A type that is a \grammarterm{template-id} in which one or more of the
\grammarterm{template-argument}{s} is an expression that references a
\grammarterm{template-parameter}.
\end{itemize}

When a type name is specified in a way that includes a nondeduced
context, all of the types that comprise that type name are also
nondeduced.
However, a compound type can include both deduced and nondeduced types.
\enterexample
If a type is specified as
\tcode{A<T>::B<T2>},
both
\tcode{T}
and
\tcode{T2}
are nondeduced.
Likewise, if a type is specified as
\tcode{A<I+J>::X<T>},
\tcode{I},
\tcode{J},
and
\tcode{T}
are nondeduced.
If a type is specified as
\tcode{void}
\tcode{f(typename}
\tcode{A<T>::B,}
\tcode{A<T>)},
the
\tcode{T}
in
\tcode{A<T>::B}
is nondeduced but
the
\tcode{T}
in
\tcode{A<T>}
is deduced.
\exitexample

\pnum
\enterexample
Here is an example in which different parameter/argument pairs produce
inconsistent template argument deductions:

\begin{codeblock}
template<class T> void f(T x, T y) { /* ... */ }
struct A { /* ... */ };
struct B : A { /* ... */ };
int g(A a, B b)
{
	f(a,b);			// error: \tcode{T} could be \tcode{A} or \tcode{B}
	f(b,a);			// error: \tcode{T} could be \tcode{A} or \tcode{B}
	f(a,a);			// OK: \tcode{T} is \tcode{A}
	f(b,b);			// OK: \tcode{T} is \tcode{B}
}
\end{codeblock}

\pnum
Here is an example where two template arguments are deduced from a
single function parameter/argument pair.
This can lead to conflicts
that cause type deduction to fail:

\begin{codeblock}
template <class T, class U> void f(  T (*)( T, U, U )  );

int g1( int, float, float);
char g2( int, float, float);
int g3( int, char, float);

void r()
{
	f(g1);			// OK: \tcode{T} is \tcode{int} and \tcode{U} is \tcode{float}
	f(g2);			// error: \tcode{T} could be \tcode{char} or \tcode{int}
	f(g3);			// error: \tcode{U} could be \tcode{char} or \tcode{float}
}
\end{codeblock}

\pnum
Here is an example where a qualification conversion applies between the
argument type on the function call and the deduced template argument type:

\begin{codeblock}
template<class T> void f(const T*) {}
int *p;
void s()
{
	f(p);			// \tcode{f(const int*)}
}
\end{codeblock}

\pnum
Here is an example where the template argument is used to instantiate
a derived class type of the corresponding function parameter type:

\begin{codeblock}
template <class T> struct B { };
template <class T> struct D : public B<T> {};
struct D2 : public B<int> {};
template <class T> void f(B<T>&){}
void t()
{
	D<int> d;
	D2     d2;
	f(d);			// calls \tcode{f(B<int>\&)}
	f(d2);			// calls \tcode{f(B<int>\&)}
}
\end{codeblock}
\exitexampleb

\pnum
A template type argument
\tcode{T},
a template template argument
\tcode{TT}
or a template non-type argument
\tcode{i}
can be deduced if
\tcode{P}
and
\tcode{A}
have one of the following forms:

\begin{codeblock}
T
@\grammarterm{cv-list}@ T
T*
T&
T[@\grammarterm{integer-constant}@]
@\grammarterm{template-name}@<T>  (where @\grammarterm{template-name}@ refers to a class template)
@\term{type}@(*)(T)
T(*)()
T(*)(T)
T @\term{type}@::*
@\term{type}@ T::*
T T::*
T (@\term{type}@::*)()
@\term{type}@ (T::*)()
@\term{type}@ (@\term{type}@::*)(T)
@\term{type}@ (T::*)(T)
T (@\term{type}@::*)(T)
T (T::*)()
T (T::*)(T)
@\term{type}@[i]
@\grammarterm{template-name}@<i>  (where @\grammarterm{template-name}@ refers to a class template)
TT<T>
TT<i>
TT<>
\end{codeblock}

where
\tcode{(T)}
represents argument lists where at least one argument type contains a
\tcode{T},
and
\tcode{()}
represents argument lists where no parameter contains a
\tcode{T}.
Similarly,
\tcode{<T>}
represents template argument lists where at least one argument contains a
\tcode{T},
\tcode{<i>}
represents template argument lists where at least one argument contains an
\tcode{i}
and
\tcode{<>}
represents template argument lists where no argument contains a
\tcode{T}
or an
\tcode{i}.

\pnum
These forms can be used in the same way as
\tcode{T}
is for further composition of types.
\enterexample

\begin{codeblock}
X<int> (*)(char[6])
\end{codeblock}

is of the form

\begin{codeblock}
@\grammarterm{template-name}@<T> (*)(@\term{type}@[i])
\end{codeblock}

which is a variant of

\begin{codeblock}
@\term{type}@ (*)(T)
\end{codeblock}

where type is
\tcode{X<int>}
and
\tcode{T}
is
\tcode{char[6]}.
\exitexample

\pnum
Template arguments cannot be deduced from function arguments involving
constructs other than the ones specified above.

\pnum
A template type argument cannot be deduced from the type of a non-type
\grammarterm{template-argument}.
\enterexample
\begin{codeblock}
template<class T, T i> void f(double a[10][i]);
int v[10][20];
f(v);				// error: argument for template-parameter \tcode{T} cannot be deduced
\end{codeblock}
\exitexampleb

\pnum
\enternote
except for reference and pointer types, a major array bound is not part of a
function parameter type and cannot be deduced from an argument:

\begin{codeblock}
template<int i> void f1(int a[10][i]);
template<int i> void f2(int a[i][20]);
template<int i> void f3(int (&a)[i][20]);

void g()
{
	int v[10][20];
	f1(v);			// OK: \tcode{i} deduced to be \tcode{20}
	f1<20>(v);		// OK
	f2(v);			// error: cannot deduce template-argument \tcode{i}
	f2<10>(v);		// OK
	f3(v);			// OK: \tcode{i} deduced to be \tcode{10}
}
\end{codeblock}

\pnum
If, in the declaration of a function template with a non-type
\grammarterm{template-parameter}, the non-type \grammarterm{template-parameter}
is used in an expression in the function parameter-list, the corresponding
\grammarterm{template-argument} must always be explicitly specified or
deduced elsewhere because type deduction would otherwise always fail
for such a \grammarterm{template-argument}.
\begin{codeblock}
template<int i> class A { /* ... */ };
template<short s> void g(A<s+1>);
void k() {
    A<1> a;
    g(a);			// error: deduction fails for expression \tcode{s+1}
    g<0>(a);			// OK
}
\end{codeblock}
\exitnoteb
\enternote
template parameters do not participate in template argument deduction if
they are used only in nondeduced contexts. For example,

\begin{codeblock}
template<int i, typename T>
T deduce(typename A<T>::X x,	// \tcode{T} is not deduced here
		  T	  t,	// but \tcode{T} is deduced here
	 typename B<i>::Y y);	// \tcode{i} is not deduced here
A<int> a;
B<77>  b;

int    x = deduce<77>(a.xm, 62, y.ym);
// \tcode{T} is deduced to be \tcode{int}, \tcode{a.xm} must be convertible to
// \tcode{A<int>::X}
// \tcode{i} is explicitly specified to be \tcode{77}, \tcode{y.ym} must be convertible
// to \tcode{B<77>::Y}
\end{codeblock}
\exitnoteb

\pnum
If, in the declaration of a function template with a non-type
\grammarterm{template-parameter,}
the non-type
\grammarterm{template-parameter}
is used in an expression in the function parameter-list and,
if the corresponding
\grammarterm{template-argument}
is deduced, the
\grammarterm{template-argument}
type shall match the type of the
\grammarterm{template-parameter}
exactly, except that a
\grammarterm{template-argument}
deduced from an array bound may be of any integral type.\footnote{Although the
\grammarterm{template-argument}
corresponding to a
\grammarterm{template-parameter}
of type
\tcode{bool}
may be deduced from an array bound, the resulting value will always be
\tcode{true}
because the array bound will be non-zero.}
\enterexample

\begin{codeblock}
template<int i> class A { /* ... */ };
template<short s> void f(A<s>);
void k1() {
    A<1> a;
    f(a);			// error: deduction fails for conversion from \tcode{int} to \tcode{short}
    f<1>(a);			// OK
}

template<const short cs> class B { };
template<short s> void h(B<s>);
void k2() {
    B<1> b;
    g(b);			// OK: cv-qualifiers are ignored on template parameter types
}
\end{codeblock}
\exitexampleb

\pnum
A
\grammarterm{template-argument}
can be deduced from a pointer to function or pointer to member function
argument if the set of overloaded functions does not contain function
templates and at most one of a set of overloaded functions provides a
unique match.
\enterexample

\begin{codeblock}
template<class T> void f(void(*)(T,int));
template<class T> void foo(T,int);
void g(int,int);
void g(char,int);

void h(int,int,int);
void h(char,int);
int m()
{
	f(&g);			// error: ambiguous
	f(&h);			// OK: void \tcode{h(char,int)} is a unique match
	f(&foo);		// error: type deduction fails because \tcode{foo} is a template
}
\end{codeblock}
\exitexampleb

\pnum
A template
\grammarterm{type-parameter}
cannot be deduced from the type of a function default argument.
\enterexample

\begin{codeblock}
template <class T> void f(T = 5, T = 7);
void g()
{
	f(1);			// OK: call \tcode{f<int>(1,7)}
	f();			// error: cannot deduce \tcode{T}
	f<int>();		// OK: call \tcode{f<int>(5,7)}
}
\end{codeblock}
\exitexampleb

\pnum
The
\grammarterm{template-argument}
corresponding to a template
\grammarterm{template-parameter}
is deduced from the type of the
\grammarterm{template-argument}
of a class template specialization used in the argument list of a function call.
\enterexample

\begin{codeblock}
template <template <class T> class X> struct A { };
template <template <class T> class X> void f(A<X>) { }
template<class T> struct B { };
A<B> ab;
f(ab);				// calls \tcode{f(A<B>)}
\end{codeblock}
\exitexampleb
\enternote a default \grammarterm{template-argument} cannot be specified
in a function template declaration or definition; therefore default
\grammarterm{template-argument}{s} cannot be used to influence template
argument deduction.
\exitnote

\rSec2[temp.over]{Overload resolution}

\pnum
\indextext{overloading!resolution!function template}%
A function template can be overloaded either by (non-template) functions of its
name or by (other) function templates of the same name.
When a call to that name is written (explicitly, or implicitly using the
operator notation), template argument deduction~(\ref{temp.deduct})
and checking of any explicit template arguments~(\ref{temp.arg}) are performed
for each function template to find the template argument values (if any) that
can be used with that function template to instantiate a function template
specialization that can be invoked with the call arguments.
For each function template, if the argument deduction and checking succeeds,
the
\grammarterm{template-argument}{s}
(deduced and/or explicit)
are used to instantiate a single function template specialization which is
added to the candidate functions set to be used in overload resolution.
If, for a given function template, argument deduction fails, no such function
is added to the set of candidate functions for that template.
The complete set of candidate functions includes all the function templates
instantiated in this way and all of the non-template overloaded functions of
the same name. The function template specializations are treated like any
other functions in
the remainder of overload resolution, except as explicitly noted
in~\ref{over.match.best}.\footnote{The parameters of function template
specializations contain no
template parameter types.
The set of conversions allowed on deduced arguments is limited, because the
argument deduction process produces function templates with parameters that
either match the call arguments exactly or differ only in ways that can be
bridged by the allowed limited conversions.
Non-deduced arguments allow the full range of conversions.
Note also that~\ref{over.match.best} specifies that a non-template function will
be given preference over a template specialization if the two functions
are otherwise equally good candidates for an overload match.}

\pnum
\enterexample
\begin{codeblock}
template<class T> T max(T a, T b) { return a>b?a:b; }

void f(int a, int b, char c, char d)
{
	int m1 = max(a,b);	// \tcode{max(int a, int b)}
	char m2 = max(c,d);	// \tcode{max(char a, char b)}
	int m3 = max(a,c);	// error: cannot generate \tcode{max(int,char)}
}
\end{codeblock}

\pnum
Adding the non-template function

\begin{codeblock}
int max(int,int);
\end{codeblock}

to the example above would resolve the third call, by providing a function that
could be called for
\tcode{max(a,c)}
after using the standard conversion of
\tcode{char}
to
\tcode{int}
for
\tcode{c}.

\pnum
Here is an example involving conversions on a function argument involved in
\grammarterm{template-argument}
deduction:

\begin{codeblock}
template<class T> struct B { /* ... */ };
template<class T> struct D : public B<T> { /* ... */ };
template<class T> void f(B<T>&);

void g(B<int>& bi, D<int>& di)
{
	f(bi);			// \tcode{f(bi)}
	f(di);			// \tcode{f( (B<int>\&)di )}
}
\end{codeblock}

\pnum
Here is an example involving conversions on a function argument not involved in
\grammarterm{template-parameter}
deduction:

\begin{codeblock}
template<class T> void f(T*,int);	// \#1
template<class T> void f(T,char);	// \#2

void h(int* pi, int i, char c)
{
	f(pi,i);		// \#1: \tcode{f<int>(pi,i)}
	f(pi,c);		// \#2: \tcode{f<int*>(pi,c)}

	f(i,c);			// \#2: \tcode{f<int>(i,c);}
	f(i,i);			// \#2: \tcode{f<int>(i,char(i))}
}
\end{codeblock}
\exitexampleb

\pnum
Only the signature of a function template specialization is needed to enter the
specialization in a set of candidate functions.
Therefore only the function template declaration is needed to resolve a call
for which a template specialization is a candidate.
\enterexample

\begin{codeblock}
template<class T> void f(T);	// declaration

void g()
{
	f("Annemarie");		// call of \tcode{f<const char*>}
}
\end{codeblock}

The call of
\tcode{f}
is well-formed even if the template
\tcode{f}
is only declared and not defined at the point of the call.
The program will be ill-formed unless a specialization for
\tcode{f<const char*>},
either implicitly or explicitly generated,
is present in some translation unit.
\exitexample%
\indextext{template|)}
